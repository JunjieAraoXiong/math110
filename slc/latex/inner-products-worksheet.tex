%==============================================================================
% EXERCISES: INNER PRODUCTS WORKSHEET
% Linear Algebra Done Right (4th ed.) - Sheldon Axler
%==============================================================================

\documentclass[11pt]{article}
\usepackage[margin=1in]{geometry}

% Required packages
\usepackage{amsmath, amssymb, amsthm}
\usepackage{enumitem}
\usepackage{fancyhdr}
\usepackage{tikz}
\usepackage{xcolor}

% Custom commands (from style guide)
\newcommand{\R}{\mathbb{R}}
\newcommand{\C}{\mathbb{C}}
\newcommand{\F}{\mathbb{F}}
\newcommand{\Z}{\mathbb{Z}}
\newcommand{\Poly}{\mathcal{P}}
\newcommand{\ip}[2]{\langle #1, #2 \rangle}
\newcommand{\norm}[1]{\|#1\|}

% Header/footer
\pagestyle{fancy}
\fancyhf{}
\fancyhead[L]{\small MATH 110}
\fancyhead[R]{\small Inner Products}
\fancyfoot[C]{\small\thepage}
\renewcommand{\headrulewidth}{0.4pt}

% Exercise counter and command
\newcounter{exercise}
\newcommand{\exercise}{\refstepcounter{exercise}\noindent\textbf{Exercise \theexercise.} }

% Ruled solution space with gray separator
\newcommand{\ruledspace}[1]{%
  \par\vspace{0.3em}%
  \noindent\textcolor{gray!30}{\hrule}%
  \vspace{#1}%
}

% Answer line for computation problems
\newcommand{\answerline}{%
  \vspace{0.5em}%
  \noindent\textbf{Answer:} \hrulefill%
  \vspace{1em}%
}

\begin{document}

\noindent
\begin{minipage}{\linewidth}
    \centering
    \textbf{\Large Inner Products} \\[0.5em]
    \textit{Linear Algebra Done Right, 4th ed.}\\[0.3em]
    \hrule
\end{minipage}
\vspace{1.5em}

%==============================================================================
% INNER PRODUCT VERIFICATION (1-2)
%==============================================================================

\exercise
Consider the space $V = \Poly_n(\C)$. Show that
\[
    \ip{f}{g} = \int_0^1 f(x) \overline{g(x)}\, dx
\]
defines an inner product on $V$.
\ruledspace{7cm}

%------------------------------------------------------------------------------

\exercise
Consider a fixed set of points $x_1, \ldots, x_n \in \C$. For two functions $f, g \in \Poly_n(\C)$, is
\[
    \ip{f}{g} = \sum_{i=1}^{n} f(x_i) \overline{g(x_i)}
\]
an inner product? For which values of $k$ is this an inner product on $\Poly_k(\C)$?
\ruledspace{7cm}

\newpage

%==============================================================================
% NORM INEQUALITIES (3-4)
%==============================================================================

\exercise
(Axler 6.A.22) Show that if $u, v \in V$, then
\[
    \norm{u + v} \cdot \norm{u - v} \leq \norm{u}^2 + \norm{v}^2.
\]
\ruledspace{7cm}

%------------------------------------------------------------------------------

\exercise
(Axler 6.A.23) Suppose $v_1, \ldots, v_m \in V$ are such that $\norm{v_k} \leq 1$ for each $k = 1, \ldots, m$. Show that there exists $a_1, \ldots, a_m \in \{1, -1\}$ such that
\[
    \norm{a_1 v_1 + \cdots + a_m v_m} \leq \sqrt{m}.
\]
\ruledspace{8cm}

\newpage

%==============================================================================
% GRAM-SCHMIDT (5)
%==============================================================================

\exercise
For $n = 1, 2, 3$, use the Gram-Schmidt algorithm to orthonormalize the basis $1, x, \ldots, x^n$ of $\Poly_n(\C)$ equipped with the inner product of Exercise 1.
\ruledspace{10cm}

\newpage

%==============================================================================
% ORTHONORMAL BASES (6-8)
%==============================================================================

\exercise
(Axler 6.B.6a) Suppose $e_1, e_2, \ldots, e_n$ is an orthonormal basis of $V$. Prove that if $v_1, v_2, \ldots, v_n$ are vectors in $V$ such that
\[
    \norm{e_k - v_k} < \frac{1}{\sqrt{n}}
\]
for each $k$, then $v_1, v_2, \ldots, v_n$ is a basis of $V$.
\ruledspace{8cm}

%------------------------------------------------------------------------------

\exercise
(Axler 6.B.18) Suppose $u_1, u_2, \ldots, u_m$ is a linearly independent list of vectors in $V$. Show that there exists $v \in V$ such that $\ip{u_i}{v} = 1$ for all $i = 1, \ldots, m$.
\ruledspace{7cm}

\newpage

%------------------------------------------------------------------------------

\exercise
Prove that every (pairwise) orthogonal list of non-zero vectors is linearly independent.
\ruledspace{6cm}

%==============================================================================
% ORTHOGONAL PROJECTIONS (9-11)
%==============================================================================

\exercise
(Axler 6.C.10) Suppose $V$ is finite-dimensional, $T \in \mathcal{L}(V)$, and $U$ is a subspace of $V$. Prove that $U$ and $U^\perp$ are both invariant under $T$ if and only if $P_U T = T P_U$.
\ruledspace{8cm}

\newpage

%------------------------------------------------------------------------------

\exercise
(Axler 6.C.12) Find $p \in \Poly_3(\R)$ such that $p(0) = 0$, $p'(0) = 0$, and
\[
    \int_0^1 |2 + 3x - p(x)|^2\, dx
\]
is as small as possible.
\ruledspace{8cm}
\answerline

%------------------------------------------------------------------------------

\exercise
(Axler 6.C.9) Suppose $V$ is finite-dimensional. Suppose $P \in \mathcal{L}(V)$ is such that $P^2 = P$ and every vector in $\operatorname{null}(P)$ is orthogonal to every vector in $\operatorname{range}(P)$. Prove that there exists a subspace $U$ of $V$ such that $P = P_U$.
\ruledspace{8cm}

\vfill

\end{document}
