%==============================================================================
% EXERCISES: VECTOR SPACES, LINEAR MAPS, AND DUALITY WORKSHEET
% Linear Algebra Done Right (4th ed.) - Sheldon Axler
%==============================================================================

\documentclass[11pt]{article}
\usepackage[margin=1in]{geometry}

% Required packages
\usepackage{amsmath, amssymb, amsthm}
\usepackage{enumitem}
\usepackage{fancyhdr}
\usepackage{tikz}
\usepackage{xcolor}

% Custom commands (from style guide)
\newcommand{\R}{\mathbb{R}}
\newcommand{\C}{\mathbb{C}}
\newcommand{\F}{\mathbb{F}}
\newcommand{\Z}{\mathbb{Z}}
\newcommand{\Poly}{\mathcal{P}}

% Header/footer
\pagestyle{fancy}
\fancyhf{}
\fancyhead[L]{\small MATH 110}
\fancyhead[R]{\small Vector Spaces, Linear Maps, and Duality}
\fancyfoot[C]{\small\thepage}
\renewcommand{\headrulewidth}{0.4pt}

% Exercise counter and command
\newcounter{exercise}
\newcommand{\exercise}{\refstepcounter{exercise}\noindent\textbf{Exercise \theexercise.} }

% Ruled solution space with gray separator
\newcommand{\ruledspace}[1]{%
  \par\vspace{0.3em}%
  \noindent\textcolor{gray!30}{\hrule}%
  \vspace{#1}%
}

% Answer line for computation problems
\newcommand{\answerline}{%
  \vspace{0.5em}%
  \noindent\textbf{Answer:} \hrulefill%
  \vspace{1em}%
}

\begin{document}

\noindent
\begin{minipage}{\linewidth}
    \centering
    \textbf{\Large Vector Spaces, Linear Maps, and Duality} \\[0.5em]
    \textit{Linear Algebra Done Right, 4th ed.}\\[0.3em]
    \hrule
\end{minipage}
\vspace{1.5em}

%==============================================================================
% SUBSPACES AND DIRECT SUMS (1)
%==============================================================================

\exercise
Given a matrix
\[
    A := \begin{bmatrix}
    a_{11} & \cdots & a_{15} \\
    \vdots & \ddots & \vdots \\
    a_{51} & \cdots & a_{55}
    \end{bmatrix} \in \R^{5 \times 5},
\]
we say that $A$ is \emph{symmetric} if and only if $a_{ij} = a_{ji}$ for each $i$ and $j$, and we say that $A$ is \emph{skew-symmetric} if and only if $a_{ij} = -a_{ji}$.

\begin{enumerate}[label=(\alph*)]
    \item Let $Y$ and $K$ be the set of symmetric and skew-symmetric matrices in $\R^{5 \times 5}$, respectively. Show that $Y$ and $K$ are subspaces of $\R^{5 \times 5}$.
    \ruledspace{6cm}

    \item Compute $\dim Y$ and $\dim K$.
    \ruledspace{5cm}
    \answerline

    \item Show that $\R^{5 \times 5}$ is the direct sum of the subspaces $Y$ and $K$.
    \ruledspace{7cm}
\end{enumerate}

\newpage

%------------------------------------------------------------------------------

\exercise
(Axler 2.C.17) Fix a finite-dimensional vector space $V$ over a field $\F$, and fix subspaces $U_1, \ldots, U_n$ of $V$.

\begin{enumerate}[label=(\alph*)]
    \item Show that
    \[
        \dim(U_1 + \cdots + U_n) \leq \dim U_1 + \cdots + \dim U_n.
    \]
    \ruledspace{6cm}

    \item In fact, show that equality holds in (a) if and only if the sum $U_1 + \cdots + U_n$ is direct.
    \ruledspace{7cm}
\end{enumerate}

%------------------------------------------------------------------------------

\exercise
Fix a linear transformation $T \colon V \to W$ of vector spaces over a field $\F$. For any $w \in W$, suppose there is a vector $v_0 \in V$ such that $T(v_0) = w$. Then show that
\[
    \{v \in V : T(v) = w\} = \{v + v_0 : v \in \operatorname{null} T\}.
\]
\ruledspace{7cm}

\newpage

%------------------------------------------------------------------------------

\exercise
(Axler 3.B.28) Suppose $p \in \Poly(\R)$. Prove that there exists a polynomial $q \in \Poly(\R)$ such that
\[
    5q'' + 3q' = p.
\]
\ruledspace{8cm}

%------------------------------------------------------------------------------

\exercise
Let $I$ be the linear map $I \colon \Poly_5(\R) \to \Poly_6(\R)$ by
\[
    I(f) := \int_0^x f(t)\, dt.
\]

\begin{enumerate}[label=(\alph*)]
    \item Convince yourself that $I$ is a linear map.
    \ruledspace{5cm}

    \item Find bases of $\Poly_5(\R)$ and $\Poly_6(\R)$.
    \ruledspace{4cm}
    \answerline

    \item Use the bases found in (b) in order to write dual bases for $\Poly_5(\R)'$ and $\Poly_6(\R)'$.
    \ruledspace{5cm}

    \item Use the dual bases found in (c) in order to write $I' \colon \Poly_6(\R)' \to \Poly_5(\R)'$ as a matrix.
    \ruledspace{6cm}
\end{enumerate}

\newpage

%------------------------------------------------------------------------------

\exercise
Let $V$ be a 2-dimensional vector space, and let $\varphi, \psi \in V'$. Show that $\varphi$ and $\psi$ are linearly independent if and only if
\[
    \operatorname{null}(\varphi) \cap \operatorname{null}(\psi) = \{0\}.
\]
\ruledspace{7cm}

%------------------------------------------------------------------------------

\exercise
Fix a finite-dimensional vector space $V$. Call a linear map $T \colon V \to V$ a \emph{projection} if and only if $T \circ T = T$.

\begin{enumerate}[label=(\alph*)]
    \item Give an example of a projection $T \colon \R^2 \to \R^2$ which is neither the zero nor the identity operators.
    \ruledspace{4cm}
    \answerline

    \item For any projection $T$, show that $T$ fixes $\operatorname{range} T$.
    \ruledspace{5cm}

    \item For any projection $T$, show that $\operatorname{range} T \cap \operatorname{null} T = \{0\}$. Conclude that $V = \operatorname{range} T \oplus \operatorname{null} T$.
    \ruledspace{6cm}

    \item We say that a linear transformation $T \in \mathcal{L}(V)$ is \emph{diagonal} if and only if there is a basis $\{v_1, \ldots, v_n\}$ of $V$ and constants $\lambda_1, \ldots, \lambda_n$ such that $Tv_i = \lambda_i v_i$ for each $i$. Show that $T$ can be written as a linear combination of projections.
    \ruledspace{7cm}
\end{enumerate}

\newpage

%------------------------------------------------------------------------------

\exercise
(Axler 3.F.23) Let $V$ be a finite-dimensional vector space, and let $U$ and $W$ be subspaces of $V$.

\begin{enumerate}[label=(\alph*)]
    \item Show that $(U + W)^0 = U^0 \cap W^0$.
    \ruledspace{7cm}

    \item Show that $(U \cap W)^0 = U^0 + W^0$.
    \ruledspace{7cm}
\end{enumerate}

\vfill

\end{document}
