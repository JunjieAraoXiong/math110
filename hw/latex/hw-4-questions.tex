\documentclass[11pt]{article}
\usepackage[margin=1in]{geometry}
\usepackage{amsmath,amssymb,amsthm}
\usepackage{fancyhdr}
\usepackage[dvipsnames]{xcolor}
\usepackage{enumitem}
\usepackage{microtype}
\usepackage{mathrsfs}

% ── Color matching the PDF ──
\definecolor{headercolor}{RGB}{0,128,128}

% ── Bold math shortcuts ──
\newcommand{\R}{\mathbf{R}}
\newcommand{\C}{\mathbf{C}}
\newcommand{\F}{\mathbf{F}}
\newcommand{\Z}{\mathbf{Z}}
\newcommand{\calL}{\mathcal{L}}
\newcommand{\calM}{\mathcal{M}}
\newcommand{\calP}{\mathcal{P}}

% ── Page style ──
\pagestyle{fancy}
\fancyhf{}
\fancyfoot[C]{Edward Frenkel}
\renewcommand{\headrulewidth}{0pt}
\renewcommand{\footrulewidth}{0pt}

% ── Exercise header command ──
% \exerciseheader{section}{number} produces the teal header line
\newcommand{\exerciseheader}[2]{%
  \newpage
  \fancyhead[L]{\textcolor{headercolor}{Instructor's Solutions Manual, Section #1}}%
  \fancyhead[R]{\textcolor{headercolor}{Exercise #2}}%
  \renewcommand{\headrulewidth}{0.4pt}%
  \renewcommand{\headrule}{\hbox to\headwidth{%
    \color{headercolor}\leaders\hrule height \headrulewidth\hfill}}%
}

% ── Exercise environment ──
\newcounter{exnum}
\newenvironment{exercise}[1]{%
  \setcounter{exnum}{#1}%
  \par\medskip\noindent\textbf{\theexnum}\quad\ignorespaces
}{\par\medskip}

\begin{document}

% ── Section 3C ──

\exerciseheader{3C}{1}
\begin{exercise}{1}
  Suppose $T \in \calL(V, W)$. Show that with respect to each choice of bases of $V$ and $W$, the matrix of $T$ has at least $\dim \operatorname{range} T$ nonzero entries.
\end{exercise}

\exerciseheader{3C}{2}
\begin{exercise}{2}
  Suppose $V$ and $W$ are finite-dimensional and $T \in \calL(V, W)$. Prove that $\dim \operatorname{range} T = 1$ if and only if there exist a basis of $V$ and a basis of $W$ such that with respect to these bases, all entries of $\calM(T)$ equal~1.
\end{exercise}

\exerciseheader{3C}{3}
\begin{exercise}{3}
  Suppose $v_1, \dots, v_n$ is a basis of $V$ and $w_1, \dots, w_m$ is a basis of $W$.
  \begin{enumerate}[label=(\alph*)]
    \item Show that if $S, T \in \calL(V, W)$, then $\calM(S + T) = \calM(S) + \calM(T)$.
    \item Show that if $\lambda \in \F$ and $T \in \calL(V, W)$, then $\calM(\lambda T) = \lambda \calM(T)$.
  \end{enumerate}

  \textit{This exercise asks you to verify 3.35 and 3.38.}
\end{exercise}

\exerciseheader{3C}{4}
\begin{exercise}{4}
  Suppose that $D \in \calL(\calP_3(\R), \calP_2(\R))$ is the differentiation map defined by $Dp = p'$. Find a basis of $\calP_3(\R)$ and a basis of $\calP_2(\R)$ such that the matrix of $D$ with respect to these bases is
  \[
    \begin{pmatrix} 1 & 0 & 0 & 0 \\ 0 & 1 & 0 & 0 \\ 0 & 0 & 1 & 0 \end{pmatrix}.
  \]

  \textit{Compare with Example 3.33. The next exercise generalizes this exercise.}
\end{exercise}

\exerciseheader{3C}{5}
\begin{exercise}{5}
  Suppose $V$ and $W$ are finite-dimensional and $T \in \calL(V, W)$. Prove that there exist a basis of $V$ and a basis of $W$ such that with respect to these bases, all entries of $\calM(T)$ are $0$ except that the entries in row $k$, column $k$, equal $1$ if $1 \le k \le \dim \operatorname{range} T$.
\end{exercise}

\exerciseheader{3C}{6}
\begin{exercise}{6}
  Suppose $v_1, \dots, v_m$ is a basis of $V$ and $W$ is finite-dimensional. Suppose $T \in \calL(V, W)$. Prove that there exists a basis $w_1, \dots, w_n$ of $W$ such that all entries in the first column of $\calM(T)$ [with respect to the bases $v_1, \dots, v_m$ and $w_1, \dots, w_n$] are $0$ except for possibly a $1$ in the first row, first column.

  \textit{In this exercise, unlike Exercise~5, you are given the basis of $V$ instead of being able to choose a basis of $V$.}
\end{exercise}

\exerciseheader{3C}{7}
\begin{exercise}{7}
  Suppose $w_1, \dots, w_n$ is a basis of $W$ and $V$ is finite-dimensional. Suppose $T \in \calL(V, W)$. Prove that there exists a basis $v_1, \dots, v_m$ of $V$ such that all entries in the first row of $\calM(T)$ [with respect to the bases $v_1, \dots, v_m$ and $w_1, \dots, w_n$] are $0$ except for possibly a $1$ in the first row, first column.

  \textit{In this exercise, unlike Exercise~5, you are given the basis of $W$ instead of being able to choose a basis of $W$.}
\end{exercise}

\exerciseheader{3C}{8}
\begin{exercise}{8}
  Suppose $A$ is an $m$-by-$n$ matrix and $B$ is an $n$-by-$p$ matrix. Prove that
  \[
    (AB)_{j,\cdot} = A_{j,\cdot}\, B
  \]
  for each $1 \le j \le m$. In other words, show that row $j$ of $AB$ equals (row $j$ of $A$) times $B$.

  \textit{This exercise gives the row version of 3.48.}
\end{exercise}

\exerciseheader{3C}{9}
\begin{exercise}{9}
  Suppose $a = \begin{pmatrix} a_1 & \cdots & a_n \end{pmatrix}$ is a $1$-by-$n$ matrix and $B$ is an $n$-by-$p$ matrix. Prove that
  \[
    aB = a_1 B_{1,\cdot} + \dots + a_n B_{n,\cdot}\,.
  \]
  In other words, show that $aB$ is a linear combination of the rows of $B$, with the scalars that multiply the rows coming from $a$.

  \textit{This exercise gives the row version of 3.50.}
\end{exercise}

\exerciseheader{3C}{10}
\begin{exercise}{10}
  Give an example of $2$-by-$2$ matrices $A$ and $B$ such that $AB \ne BA$.
\end{exercise}

\exerciseheader{3C}{11}
\begin{exercise}{11}
  Prove that the distributive property holds for matrix addition and matrix multiplication. In other words, suppose $A$, $B$, $C$, $D$, $E$, and $F$ are matrices whose sizes are such that $A(B + C)$ and $(D + E)F$ make sense. Explain why $AB + AC$ and $DF + EF$ both make sense and prove that
  \[
    A(B + C) = AB + AC \quad \text{and} \quad (D + E)F = DF + EF.
  \]
\end{exercise}

\exerciseheader{3C}{12}
\begin{exercise}{12}
  Prove that matrix multiplication is associative. In other words, suppose $A$, $B$, and $C$ are matrices whose sizes are such that $(AB)C$ makes sense. Explain why $A(BC)$ makes sense and prove that
  \[
    (AB)C = A(BC).
  \]

  \textit{Try to find a clean proof that illustrates the following quote from Emil Artin: ``It is my experience that proofs involving matrices can be shortened by 50\% if one throws the matrices out.''}
\end{exercise}

\exerciseheader{3C}{13}
\begin{exercise}{13}
  Suppose $A$ is an $n$-by-$n$ matrix and $1 \le j, k \le n$. Show that the entry in row $j$, column $k$, of $A^3$ (which is defined to mean $AAA$) is
  \[
    \sum_{p=1}^{n} \sum_{r=1}^{n} A_{j,p}\, A_{p,r}\, A_{r,k}.
  \]
\end{exercise}

\exerciseheader{3C}{14}
\begin{exercise}{14}
  Suppose $m$ and $n$ are positive integers. Prove that the function $A \mapsto A^\mathsf{t}$ is a linear map from $\F^{m,n}$ to $\F^{n,m}$.
\end{exercise}

\exerciseheader{3C}{15}
\begin{exercise}{15}
  Prove that if $A$ is an $m$-by-$n$ matrix and $C$ is an $n$-by-$p$ matrix, then
  \[
    (AC)^\mathsf{t} = C^\mathsf{t} A^\mathsf{t}.
  \]

  \textit{This exercise shows that the transpose of the product of two matrices is the product of the transposes in the opposite order.}
\end{exercise}

\exerciseheader{3C}{16}
\begin{exercise}{16}
  Suppose $A$ is an $m$-by-$n$ matrix with $A \ne 0$. Prove that the rank of $A$ is $1$ if and only if there exist $(c_1, \dots, c_m) \in \F^m$ and $(d_1, \dots, d_n) \in \F^n$ such that $A_{j,k} = c_j d_k$ for every $j = 1, \dots, m$ and every $k = 1, \dots, n$.
\end{exercise}

\exerciseheader{3C}{17}
\begin{exercise}{17}
  Suppose $T \in \calL(V)$, and $u_1, \dots, u_n$ and $v_1, \dots, v_n$ are bases of $V$. Prove that the following are equivalent.
  \begin{enumerate}[label=(\alph*)]
    \item $T$ is injective.
    \item The columns of $\calM(T)$ are linearly independent in $\F^{n,1}$.
    \item The columns of $\calM(T)$ span $\F^{n,1}$.
    \item The rows of $\calM(T)$ span $\F^{1,n}$.
    \item The rows of $\calM(T)$ are linearly independent in $\F^{1,n}$.
  \end{enumerate}

  Here $\calM(T)$ means $\calM(T, (u_1, \dots, u_n), (v_1, \dots, v_n))$.
\end{exercise}

% ── Section 3F ──

\exerciseheader{3F}{1}
\begin{exercise}{1}
  Explain why each linear functional is surjective or is the zero map.
\end{exercise}

\exerciseheader{3F}{2}
\begin{exercise}{2}
  Give three distinct examples of linear functionals on $\R^{[0,1]}$.
\end{exercise}

\exerciseheader{3F}{3}
\begin{exercise}{3}
  Suppose $V$ is finite-dimensional and $v \in V$ with $v \ne 0$. Prove that there exists $\varphi \in V'$ such that $\varphi(v) = 1$.
\end{exercise}

\exerciseheader{3F}{4}
\begin{exercise}{4}
  Suppose $V$ is finite-dimensional and $U$ is a subspace of $V$ such that $U \ne V$. Prove that there exists $\varphi \in V'$ such that $\varphi(u) = 0$ for every $u \in U$ but $\varphi \ne 0$.
\end{exercise}

\exerciseheader{3F}{5}
\begin{exercise}{5}
  Suppose $T \in \calL(V, W)$ and $w_1, \dots, w_m$ is a basis of $\operatorname{range} T$. Hence for each $v \in V$, there exist unique numbers $\varphi_1(v), \dots, \varphi_m(v)$ such that
  \[
    Tv = \varphi_1(v) w_1 + \dots + \varphi_m(v) w_m,
  \]
  thus defining functions $\varphi_1, \dots, \varphi_m$ from $V$ to $\F$. Show that each of the functions $\varphi_1, \dots, \varphi_m$ is a linear functional on $V$.
\end{exercise}

\exerciseheader{3F}{6}
\begin{exercise}{6}
  Suppose $\varphi, \beta \in V'$. Prove that $\operatorname{null} \varphi \subseteq \operatorname{null} \beta$ if and only if there exists $c \in \F$ such that $\beta = c\varphi$.
\end{exercise}

\exerciseheader{3F}{7}
\begin{exercise}{7}
  Suppose that $V_1, \dots, V_m$ are vector spaces. Prove that $(V_1 \times \dots \times V_m)'$ and $V_1' \times \dots \times V_m'$ are isomorphic vector spaces.
\end{exercise}

\exerciseheader{3F}{8}
\begin{exercise}{8}
  Suppose $v_1, \dots, v_n$ is a basis of $V$ and $\varphi_1, \dots, \varphi_n$ is the dual basis of $V'$. Define $\Gamma \colon V \to \F^n$ and $\Lambda \colon \F^n \to V$ by
  \[
    \Gamma(v) = (\varphi_1(v), \dots, \varphi_n(v)) \quad \text{and} \quad \Lambda(a_1, \dots, a_n) = a_1 v_1 + \dots + a_n v_n.
  \]
  Explain why $\Gamma$ and $\Lambda$ are inverses of each other.
\end{exercise}

\exerciseheader{3F}{9}
\begin{exercise}{9}
  Suppose $m$ is a positive integer. Show that the dual basis of the basis $1, x, \dots, x^m$ of $\calP_m(\R)$ is $\varphi_0, \varphi_1, \dots, \varphi_m$, where
  \[
    \varphi_k(p) = \frac{p^{(k)}(0)}{k!}.
  \]

  \textit{Here $p^{(k)}$ denotes the $k^\text{th}$ derivative of $p$, with the understanding that the $0^\text{th}$ derivative of $p$ is $p$.}
\end{exercise}

\exerciseheader{3F}{10}
\begin{exercise}{10}
  Suppose $m$ is a positive integer.
  \begin{enumerate}[label=(\alph*)]
    \item Show that $1, x - 5, \dots, (x - 5)^m$ is a basis of $\calP_m(\R)$.
    \item What is the dual basis of the basis in~(a)?
  \end{enumerate}
\end{exercise}

\exerciseheader{3F}{11}
\begin{exercise}{11}
  Suppose $v_1, \dots, v_n$ is a basis of $V$ and $\varphi_1, \dots, \varphi_n$ is the corresponding dual basis of $V'$. Suppose $\psi \in V'$. Prove that
  \[
    \psi = \psi(v_1)\varphi_1 + \dots + \psi(v_n)\varphi_n.
  \]
\end{exercise}

\exerciseheader{3F}{12}
\begin{exercise}{12}
  Suppose $S, T \in \calL(V, W)$.
  \begin{enumerate}[label=(\alph*)]
    \item Prove that $(S + T)' = S' + T'$.
    \item Prove that $(\lambda T)' = \lambda T'$ for all $\lambda \in \F$.
  \end{enumerate}

  \textit{This exercise asks you to verify (a) and (b) in 3.120.}
\end{exercise}

\exerciseheader{3F}{13}
\begin{exercise}{13}
  Show that the dual map of the identity operator on $V$ is the identity operator on $V'$.
\end{exercise}

\exerciseheader{3F}{14}
\begin{exercise}{14}
  Define $T \colon \R^3 \to \R^2$ by
  \[
    T(x, y, z) = (4x + 5y + 6z,\; 7x + 8y + 9z).
  \]
  Suppose $\varphi_1, \varphi_2$ denotes the dual basis of the standard basis of $\R^2$ and $\psi_1, \psi_2, \psi_3$ denotes the dual basis of the standard basis of $\R^3$.
  \begin{enumerate}[label=(\alph*)]
    \item Describe the linear functionals $T'(\varphi_1)$ and $T'(\varphi_2)$.
    \item Write $T'(\varphi_1)$ and $T'(\varphi_2)$ as linear combinations of $\psi_1, \psi_2, \psi_3$.
  \end{enumerate}
\end{exercise}

\exerciseheader{3F}{15}
\begin{exercise}{15}
  Define $T \colon \calP(\R) \to \calP(\R)$ by
  \[
    (Tp)(x) = x^2 p(x) + p''(x)
  \]
  for each $x \in \R$.
  \begin{enumerate}[label=(\alph*)]
    \item Suppose $\varphi \in \calP(\R)'$ is defined by $\varphi(p) = p'(4)$. Describe the linear functional $T'(\varphi)$ on $\calP(\R)$.
    \item Suppose $\varphi \in \calP(\R)'$ is defined by $\varphi(p) = \int_0^1 p$. Evaluate $(T'(\varphi))(x^3)$.
  \end{enumerate}
\end{exercise}

\exerciseheader{3F}{16}
\begin{exercise}{16}
  Suppose $W$ is finite-dimensional and $T \in \calL(V, W)$. Prove that
  \[
    T' = 0 \iff T = 0.
  \]
\end{exercise}

\exerciseheader{3F}{17}
\begin{exercise}{17}
  Suppose $V$ and $W$ are finite-dimensional and $T \in \calL(V, W)$. Prove that $T$ is invertible if and only if $T' \in \calL(W', V')$ is invertible.
\end{exercise}

\exerciseheader{3F}{18}
\begin{exercise}{18}
  Suppose $V$ and $W$ are finite-dimensional. Prove that the map that takes $T \in \calL(V, W)$ to $T' \in \calL(W', V')$ is an isomorphism of $\calL(V, W)$ onto $\calL(W', V')$.
\end{exercise}

\end{document}
