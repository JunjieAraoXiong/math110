% ── Section 3F (Exercises 1--18) ──

\exerciseheader{3F}{1}
\begin{exercise}{1}
  Explain why each linear functional is surjective or is the zero map.
\end{exercise}
\begin{solution}
  A linear functional is a linear map into $\F$. The range of a linear map is a subspace of the target space. The only subspaces of $\F$ are $\F$ and $\{0\}$. Thus each linear functional is surjective (if the range equals $\F$) or is the zero map (if the range equals $\{0\}$).
\end{solution}

\exerciseheader{3F}{2}
\begin{exercise}{2}
  Give three distinct examples of linear functionals on $\R^{[0,1]}$.
\end{exercise}
\begin{solution}
  For $t \in [0,1]$, define $\varphi_t \colon \R^{[0,1]} \to \R$ by
  \[
    \varphi_t(f) = f(t)
  \]
  for each $f \in \R^{[0,1]}$.

  For each $t \in [0,1]$, $\varphi_t$ is a linear functional on $\R^{[0,1]}$. Thus three distinct linear functionals on $\R^{[0,1]}$ are $\varphi_0$, $\varphi_{1/2}$, and $\varphi_1$.
\end{solution}

\exerciseheader{3F}{3}
\begin{exercise}{3}
  Suppose $V$ is finite-dimensional and $v \in V$ with $v \ne 0$. Prove that there exists $\varphi \in V'$ such that $\varphi(v) = 1$.
\end{exercise}
\begin{solution}
  Because $v \ne 0$, we can extend $v$ to a basis $v, v_2, \dots, v_n$ of $V$ (by 2.32). Now the linear map lemma (3.4) implies that there exists $\varphi \in V'$ such that $\varphi(v) = 1$ [and $\varphi(v_k)$ equals whatever we want for each $k = 2, \dots, n$].
\end{solution}

\exerciseheader{3F}{4}
\begin{exercise}{4}
  Suppose $V$ is finite-dimensional and $U$ is a subspace of $V$ such that $U \ne V$. Prove that there exists $\varphi \in V'$ such that $\varphi(u) = 0$ for every $u \in U$ but $\varphi \ne 0$.
\end{exercise}
\begin{solution}
  Let $u_1, \dots, u_m$ be a basis of $U$. Extend this list to a basis
  \[
    u_1, \dots, u_m, v_1, \dots, v_n
  \]
  of $V$ (using 2.32). Now the linear map lemma (3.4) implies that there exists $\varphi \in V'$ such that
  \[
    \varphi(u_k) = 0 \text{ for } k = 1, \dots, m
  \]
  and
  \[
    \varphi(v_k) = 1 \text{ for } k = 1, \dots, n.
  \]
  Thus $\varphi(u) = 0$ for every $u \in U$ but $\varphi \ne 0$, as desired.
\end{solution}

\exerciseheader{3F}{5}
\begin{exercise}{5}
  Suppose $T \in \calL(V, W)$ and $w_1, \dots, w_m$ is a basis of $\operatorname{range} T$. Hence for each $v \in V$, there exist unique numbers $\varphi_1(v), \dots, \varphi_m(v)$ such that
  \[
    Tv = \varphi_1(v) w_1 + \dots + \varphi_m(v) w_m,
  \]
  thus defining functions $\varphi_1, \dots, \varphi_m$ from $V$ to $\F$. Show that each of the functions $\varphi_1, \dots, \varphi_m$ is a linear functional on $V$.
\end{exercise}
\begin{solution}
  Suppose $u, v \in V$. Then
  \begin{align*}
    \varphi_1(u + v) w_1 + \dots + \varphi_m(u + v) w_m &= T(u + v) \\
      &= Tu + Tv \\
      &= (\varphi_1(u) w_1 + \dots + \varphi_m(u) w_m) + (\varphi_1(v) w_1 + \dots + \varphi_m(v) w_m) \\
      &= (\varphi_1(u) + \varphi_1(v)) w_1 + \dots + (\varphi_m(u) + \varphi_m(v)) w_m.
  \end{align*}
  Because $w_1, \dots, w_m$ is linearly independent, the equation above shows that
  \[
    \varphi_1(u + v) = \varphi_1(u) + \varphi_1(v), \;\dots,\; \varphi_m(u + v) = \varphi_m(u) + \varphi_m(v).
  \]

  Similarly, suppose now that $\lambda \in \F$ and $v \in V$. Then
  \begin{align*}
    \varphi_1(\lambda v) w_1 + \dots + \varphi_m(\lambda v) w_m &= T(\lambda v) \\
      &= \lambda Tv \\
      &= \lambda(\varphi_1(v) w_1 + \dots + \varphi_m(v) w_m) \\
      &= \lambda \varphi_1(v) w_1 + \dots + \lambda \varphi_m(v) w_m.
  \end{align*}
  Because $w_1, \dots, w_m$ is linearly independent, the equation above shows that
  \[
    \varphi_1(\lambda v) = \lambda \varphi_1(v), \;\dots,\; \varphi_m(\lambda v) = \lambda \varphi_m(v),
  \]
  completing the proof that each of the functions $\varphi_1, \dots, \varphi_m$ is a linear functional on $V$.
\end{solution}

\exerciseheader{3F}{6}
\begin{exercise}{6}
  Suppose $\varphi, \beta \in V'$. Prove that $\operatorname{null} \varphi \subseteq \operatorname{null} \beta$ if and only if there exists $c \in \F$ such that $\beta = c\varphi$.
\end{exercise}
\begin{solution}
  If $\beta = c\varphi$ for some $c \in \F$, then clearly $\operatorname{null} \varphi \subseteq \operatorname{null} \beta$.

  The implication in the other direction follows from Exercise~25 in Section~3B (with $W = \F$) and from Exercise~7 in Section~3A.
\end{solution}

\exerciseheader{3F}{7}
\begin{exercise}{7}
  Suppose that $V_1, \dots, V_m$ are vector spaces. Prove that $(V_1 \times \dots \times V_m)'$ and $V_1' \times \dots \times V_m'$ are isomorphic vector spaces.
\end{exercise}
\begin{solution}
  For $k = 1, \dots, m$ and $\varphi \in (V_1 \times \dots \times V_m)'$, define $\varphi_k \in V_k'$ by
  \[
    \varphi_k(v) = \varphi(0, \dots, 0, v, 0, \dots, 0),
  \]
  for each $v \in V_k$, where the $v$ on the right side above appears in the $k^\text{th}$ slot.

  Now define $\Gamma \colon (V_1 \times \dots \times V_m)' \to V_1' \times \dots \times V_m'$ by
  \[
    \Gamma\varphi = (\varphi_1, \dots, \varphi_m).
  \]
  It is easy to verify that $\Gamma$ is a linear map and that $\Gamma$ is injective and surjective. Thus $(V_1 \times \dots \times V_m)'$ and $V_1' \times \dots \times V_m'$ are isomorphic vector spaces.
\end{solution}

\exerciseheader{3F}{8}
\begin{exercise}{8}
  Suppose $v_1, \dots, v_n$ is a basis of $V$ and $\varphi_1, \dots, \varphi_n$ is the dual basis of $V'$. Define $\Gamma \colon V \to \F^n$ and $\Lambda \colon \F^n \to V$ by
  \[
    \Gamma(v) = (\varphi_1(v), \dots, \varphi_n(v)) \quad \text{and} \quad \Lambda(a_1, \dots, a_n) = a_1 v_1 + \dots + a_n v_n.
  \]
  Explain why $\Gamma$ and $\Lambda$ are inverses of each other.
\end{exercise}
\begin{solution}
  First suppose $a_1, \dots, a_n \in \F^n$. Then
  \begin{align*}
    \Gamma(\Lambda(a_1, \dots, a_n)) &= \Gamma(a_1 v_1 + \dots + a_n v_n) \\
      &= (\varphi_1(a_1 v_1 + \dots + a_n v_n), \dots, \varphi_n(a_1 v_1 + \dots + a_n v_n)) \\
      &= (a_1, \dots, a_n).
  \end{align*}

  Also,
  \begin{align*}
    \Lambda(\Gamma(v)) &= \Lambda(\varphi_1(v), \dots, \varphi_n(v)) \\
      &= \varphi_1(v) v_1 + \dots + \varphi_n(v) v_n \\
      &= v,
  \end{align*}
  where the last line follows from 3.114.
\end{solution}

\exerciseheader{3F}{9}
\begin{exercise}{9}
  Suppose $m$ is a positive integer. Show that the dual basis of the basis $1, x, \dots, x^m$ of $\calP_m(\R)$ is $\varphi_0, \varphi_1, \dots, \varphi_m$, where
  \[
    \varphi_k(p) = \frac{p^{(k)}(0)}{k!}.
  \]

  \textit{Here $p^{(k)}$ denotes the $k^\text{th}$ derivative of $p$, with the understanding that the $0^\text{th}$ derivative of $p$ is $p$.}
\end{exercise}
\begin{solution}
  Note that
  \[
    \varphi_k(x^j) = \frac{(x^j)^{(k)}(0)}{k!} = \begin{cases} 1 & \text{if } j = k, \\ 0 & \text{if } j \ne k. \end{cases}
  \]
  Thus $\varphi_0, \varphi_1, \dots, \varphi_m$ is indeed the dual basis of $1, x, \dots, x^m$.
\end{solution}

\exerciseheader{3F}{10}
\begin{exercise}{10}
  Suppose $m$ is a positive integer.
  \begin{enumerate}[label=(\alph*)]
    \item Show that $1, x - 5, \dots, (x - 5)^m$ is a basis of $\calP_m(\R)$.
    \item What is the dual basis of the basis in~(a)?
  \end{enumerate}
\end{exercise}
\begin{solution}
  \begin{enumerate}[label=(\alph*)]
    \item Define $\varphi_0, \varphi_1, \dots, \varphi_m \in (\calP_m(\R))'$ by
    \[
      \varphi_j(p) = \frac{p^{(j)}(5)}{j!}.
    \]
    Suppose $a_0, a_1, \dots, a_m \in \R$ and
    \[
      a_0 + a_1(x - 5) + \dots + a_m(x - 5)^m = 0.
    \]
    Then for each $j = 0, 1, \dots, m$, we have
    \[
      a_j = \varphi_j(a_0 + a_1(x - 5) + \dots + a_m(x - 5)^m) = \varphi_j(0) = 0.
    \]
    Thus $a_0 = a_1 = \dots = a_m = 0$. Hence $1, x - 5, \dots, (x - 5)^m$ is a linearly independent list in $\calP_m(\R)$ of length $m + 1$, which equals the dimension of $\calP_m(\R)$. Thus $1, x - 5, \dots, (x - 5)^m$ is a basis of $\calP_m(\R)$ (by 2.38).

    \item Let $\varphi_0, \varphi_1, \dots, \varphi_m \in (\calP_m(\R))'$ be defined as in~(a). Then
    \[
      \varphi_j((x - 5)^k) = \begin{cases} 1 & \text{if } k = j, \\ 0 & \text{if } k \ne j. \end{cases}
    \]
    Thus $\varphi_0, \varphi_1, \dots, \varphi_m$ is the dual basis of $1, x - 5, \dots, (x - 5)^m$.
  \end{enumerate}
\end{solution}

\exerciseheader{3F}{11}
\begin{exercise}{11}
  Suppose $v_1, \dots, v_n$ is a basis of $V$ and $\varphi_1, \dots, \varphi_n$ is the corresponding dual basis of $V'$. Suppose $\psi \in V'$. Prove that
  \[
    \psi = \psi(v_1)\varphi_1 + \dots + \psi(v_n)\varphi_n.
  \]
\end{exercise}
\begin{solution}
  We have
  \[
    (\psi(v_1)\varphi_1 + \dots + \psi(v_n)\varphi_n)(v_k) = \psi(v_k)
  \]
  for each $k = 1, \dots, n$. Because the linear functionals $\psi(v_1)\varphi_1 + \dots + \psi(v_n)\varphi_n$ and $\psi$ agree on a basis, they are equal by the uniqueness part of the linear map lemma (3.4).
\end{solution}

\exerciseheader{3F}{12}
\begin{exercise}{12}
  Suppose $S, T \in \calL(V, W)$.
  \begin{enumerate}[label=(\alph*)]
    \item Prove that $(S + T)' = S' + T'$.
    \item Prove that $(\lambda T)' = \lambda T'$ for all $\lambda \in \F$.
  \end{enumerate}

  \textit{This exercise asks you to verify (a) and (b) in 3.120.}
\end{exercise}
\begin{solution}
  \begin{enumerate}[label=(\alph*)]
    \item We have
    \begin{align*}
      (S + T)'(\varphi) &= \varphi \circ (S + T) \\
        &= \varphi \circ S + \varphi \circ T \\
        &= S'(\varphi) + T'(\varphi).
    \end{align*}
    Thus $(S + T)' = S' + T'$, as desired.

    \item Suppose $\lambda \in \F$. Then
    \begin{align*}
      (\lambda T)'(\varphi) &= \varphi \circ (\lambda T) \\
        &= \lambda(\varphi \circ T) \\
        &= \lambda T'(\varphi).
    \end{align*}
    Thus $(\lambda T)' = \lambda T'$, as desired.
  \end{enumerate}
\end{solution}

\exerciseheader{3F}{13}
\begin{exercise}{13}
  Show that the dual map of the identity operator on $V$ is the identity operator on $V'$.
\end{exercise}
\begin{solution}
  Let $I$ denote the identity operator on $V$. For $\varphi \in V'$ we have
  \[
    I'(\varphi) = \varphi \circ I = \varphi.
  \]
  Thus $I'$ is the identity operator on $V'$.
\end{solution}

\exerciseheader{3F}{14}
\begin{exercise}{14}
  Define $T \colon \R^3 \to \R^2$ by
  \[
    T(x, y, z) = (4x + 5y + 6z,\; 7x + 8y + 9z).
  \]
  Suppose $\varphi_1, \varphi_2$ denotes the dual basis of the standard basis of $\R^2$ and $\psi_1, \psi_2, \psi_3$ denotes the dual basis of the standard basis of $\R^3$.
  \begin{enumerate}[label=(\alph*)]
    \item Describe the linear functionals $T'(\varphi_1)$ and $T'(\varphi_2)$.
    \item Write $T'(\varphi_1)$ and $T'(\varphi_2)$ as linear combinations of $\psi_1, \psi_2, \psi_3$.
  \end{enumerate}
\end{exercise}
\begin{solution}
  \begin{enumerate}[label=(\alph*)]
    \item Note that $\varphi_1(a, b) = a$ and $\varphi_2(a, b) = b$ for all $(a, b) \in \R^2$. Recall that $T'(\varphi_1) = \varphi_1 \circ T$ and $T'(\varphi_2) = \varphi_2 \circ T$. Thus
    \[
      (T'(\varphi_1))(x, y, z) = 4x + 5y + 6z
    \]
    and
    \[
      (T'(\varphi_2))(x, y, z) = 7x + 8y + 9z.
    \]

    \item We have $\psi_1(x, y, z) = x$, $\psi_2(x, y, z) = y$, and $\psi_3(x, y, z) = z$. Using~(a), we thus see that
    \[
      T'(\varphi_1) = 4\psi_1 + 5\psi_2 + 6\psi_3
    \]
    and
    \[
      T'(\varphi_2) = 7\psi_1 + 8\psi_2 + 9\psi_3.
    \]
  \end{enumerate}
\end{solution}

\exerciseheader{3F}{15}
\begin{exercise}{15}
  Define $T \colon \calP(\R) \to \calP(\R)$ by
  \[
    (Tp)(x) = x^2 p(x) + p''(x)
  \]
  for each $x \in \R$.
  \begin{enumerate}[label=(\alph*)]
    \item Suppose $\varphi \in \calP(\R)'$ is defined by $\varphi(p) = p'(4)$. Describe the linear functional $T'(\varphi)$ on $\calP(\R)$.
    \item Suppose $\varphi \in \calP(\R)'$ is defined by $\varphi(p) = \int_0^1 p$. Evaluate $(T'(\varphi))(x^3)$.
  \end{enumerate}
\end{exercise}
\begin{solution}
  \begin{enumerate}[label=(\alph*)]
    \item Suppose $p \in \calP(\R)$. Then
    \begin{align*}
      (T'(\varphi))(p) &= (\varphi \circ T)(p) \\
        &= \varphi(Tp) \\
        &= \varphi(x^2 p + p'') \\
        &= (x^2 p + p'')'(4) \\
        &= (2xp + x^2 p' + p''')(4) \\
        &= 8p(4) + 16p'(4) + p'''(4).
    \end{align*}

    \item We have
    \begin{align*}
      (T'(\varphi))(x^3) &= (\varphi \circ T)(x^3) \\
        &= \varphi(Tx^3) \\
        &= \varphi(x^5 + 6x) \\
        &= \int_0^1 (x^5 + 6x)\, dx \\
        &= \frac{19}{6}.
    \end{align*}
  \end{enumerate}
\end{solution}

\exerciseheader{3F}{16}
\begin{exercise}{16}
  Suppose $W$ is finite-dimensional and $T \in \calL(V, W)$. Prove that
  \[
    T' = 0 \iff T = 0.
  \]
\end{exercise}
\begin{solution}
  First suppose $T = 0$. If $\varphi \in W'$, then $T'(\varphi) = \varphi \circ T = 0$, and thus $T' = 0$.

  To prove the other direction, now suppose $T' = 0$. Thus
  \[
    0 = T'(\varphi) = \varphi \circ T
  \]
  for every $\varphi \in W'$. Thus
  \[
    0 = \varphi(Tv)
  \]
  for every $v \in V$ and every $\varphi \in W'$. Exercise~3 now implies that $Tv = 0$ for every $v \in V$. Thus $T = 0$, as desired.
\end{solution}

\exerciseheader{3F}{17}
\begin{exercise}{17}
  Suppose $V$ and $W$ are finite-dimensional and $T \in \calL(V, W)$. Prove that $T$ is invertible if and only if $T' \in \calL(W', V')$ is invertible.
\end{exercise}
\begin{solution}
  A linear map is invertible if and only if it is injective and surjective. Thus the desired result follows from 3.129 and 3.131.
\end{solution}

\exerciseheader{3F}{18}
\begin{exercise}{18}
  Suppose $V$ and $W$ are finite-dimensional. Prove that the map that takes $T \in \calL(V, W)$ to $T' \in \calL(W', V')$ is an isomorphism of $\calL(V, W)$ onto $\calL(W', V')$.
\end{exercise}
\begin{solution}
  Define $\Gamma \colon \calL(V, W) \to \calL(W', V')$ by $\Gamma(T) = T'$. By 3.120, $\Gamma$ is a linear map.

  Exercise~16 implies that $\Gamma$ is injective. Note that
  \begin{align*}
    \dim \calL(V, W) &= (\dim V)(\dim W) \\
      &= (\dim V')(\dim W') \\
      &= (\dim W')(\dim V') \\
      &= \dim \calL(W', V'),
  \end{align*}
  where the first and last equalities above come from 3.72 and the second equality above comes from 3.111.

  The fundamental theorem of linear maps (3.21) and the equation above now imply that $\dim \operatorname{range} \Gamma = \dim(\calL(W', V'))$. Thus $\Gamma$ is surjective and hence is an isomorphism of $\calL(V, W)$ onto $\calL(W', V')$, as desired.
\end{solution}
