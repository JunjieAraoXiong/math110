% ── Section 1A ──

\exerciseheader{1A}{1}
\begin{exercise}{1}
  Show that $\alpha + \beta = \beta + \alpha$ for all $\alpha, \beta \in \C$.
\end{exercise}
\begin{solution}
  Suppose
  \[
    \alpha = a + bi \quad \text{and} \quad \beta = c + di,
  \]
  where $a, b, c, d \in \R$. Then
  \begin{align*}
    \alpha + \beta &= (a + bi) + (c + di) \\
                   &= (a + c) + (b + d)i \\
                   &= (c + a) + (d + b)i \\
                   &= (c + di) + (a + bi) \\
                   &= \beta + \alpha,
  \end{align*}
  where the second and fourth equalities hold because of the definition of addition in $\C$ and the third equality holds because of addition is commutative on $\R$.
\end{solution}

\exerciseheader{1A}{5}
\begin{exercise}{5}
  Show that for every $\alpha \in \C$, there exists a unique $\beta \in \C$ such that $\alpha + \beta = 0$.
\end{exercise}
\begin{solution}
  Suppose $\alpha = a + bi$, where $a, b \in \R$. Let $\beta = -a - bi$. Then the definition of complex addition shows that
  \[
    \alpha + \beta = 0.
  \]

  Suppose $\lambda \in \C$ is such that
  \[
    \alpha + \lambda = 0.
  \]
  Adding $\beta$ to both sides of the equation above shows that $\lambda = \beta$. Thus $\alpha$ has a unique additive inverse.
\end{solution}

\exerciseheader{1A}{6}
\begin{exercise}{6}
  Show that for every $\alpha \in \C$ with $\alpha \neq 0$, there exists a unique $\beta \in \C$ such that $\alpha\beta = 1$.
\end{exercise}
\begin{solution}
  Suppose $\alpha = a + bi$, where $a, b \in \R$ with at least one of $a, b$ not equal to~0. Suppose $c, d \in \R$ are such that
  \[
    (a + bi)(c + di) = 1.
  \]
  Multiply both sides of the equation above by $a - bi$, getting
  \[
    (a^2 + b^2)(c + di) = a - bi.
  \]
  Thus
  \[
    (a^2 + b^2)c = a \quad \text{and} \quad (a^2 + b^2)d = -b,
  \]
  which implies that
  \[
    c = \frac{a}{a^2 + b^2} \quad \text{and} \quad d = \frac{-b}{a^2 + b^2}.
  \]
  The equations above show that there is at most one $\beta \in \C$ such that $\alpha\beta = 1$.

  The equations above motivate us to define $\beta \in \C$ by
  \[
    \beta = \frac{a}{a^2 + b^2} - \frac{b}{a^2 + b^2}i.
  \]
  The definition of complex multiplication shows that
  \[
    \alpha\beta = 1.
  \]
\end{solution}

\exerciseheader{1A}{7}
\begin{exercise}{7}
  Show that
  \[
    \frac{-1 + \sqrt{3}\,i}{2}
  \]
  is a cube root of 1 (meaning that its cube equals 1).
\end{exercise}
\begin{solution}
  Using the definition of complex multiplication, we have
  \[
    \Bigl(\frac{-1 + \sqrt{3}\,i}{2}\Bigr)^2 = \frac{-1 - \sqrt{3}\,i}{2}.
  \]
  Thus
  \begin{align*}
    \Bigl(\frac{-1 + \sqrt{3}\,i}{2}\Bigr)^3
      &= \Bigl(\frac{-1 - \sqrt{3}\,i}{2}\Bigr)\Bigl(\frac{-1 + \sqrt{3}\,i}{2}\Bigr) \\
      &= 1.
  \end{align*}
\end{solution}

\exerciseheader{1A}{8}
\begin{exercise}{8}
  Find two distinct square roots of $i$.
\end{exercise}
\begin{solution}
  Suppose $a$ and $b$ are real numbers such that
  \[
    (a + bi)^2 = i.
  \]
  Then
  \begin{align*}
    i &= (a + bi)^2 \\
      &= (a^2 - b^2) + 2abi.
  \end{align*}
  Thus
  \[
    a^2 = b^2 \quad \text{and} \quad 2ab = 1.
  \]

  The equation $a^2 = b^2$ implies that $a = b$ or $a = -b$. However, if $a = -b$, the equation $2ab = 1$ implies that $-2b^2 = 1$, which is impossible because $b$ is a real number.

  Thus we have $a = b$. The equation $2ab = 1$ now becomes the equation $2b^2 = 1$, which leads to $a = b = \pm\frac{\sqrt{2}}{2}$.

  Hence the only two possibilities for square roots of $i$ are
  \[
    \frac{\sqrt{2}}{2} + \frac{\sqrt{2}}{2}\,i \quad \text{and} \quad -\frac{\sqrt{2}}{2} - \frac{\sqrt{2}}{2}\,i.
  \]
  Squaring each of the numbers above gives $i$. Thus the two numbers above are indeed square roots of $i$.
\end{solution}

\exerciseheader{1A}{10}
\begin{exercise}{10}
  Explain why there does not exist $\lambda \in \C$ such that
  \[
    \lambda(2 - 3i, 5 + 4i, -6 + 7i) = (12 - 5i, 7 + 22i, -32 - 9i).
  \]
\end{exercise}
\begin{solution}
  The equation above is equivalent to the equation
  \[
    \bigl(\lambda(2 - 3i), \lambda(5 + 4i), \lambda(-6 + 7i)\bigr) = (12 - 5i, 7 + 22i, -32 - 9i),
  \]
  which is equivalent to the three equations
  \[
    \lambda(2 - 3i) = 12 - 5i, \quad \lambda(5 + 4i) = 7 + 22i, \quad \lambda(-6 + 7i) = -32 - 9i.
  \]
  The first equation above implies that
  \begin{align*}
    \lambda &= \frac{12 - 5i}{2 - 3i} \\[6pt]
            &= \frac{12 - 5i}{2 - 3i} \cdot \frac{2 + 3i}{2 + 3i} \\[6pt]
            &= \frac{(24 + 15) + (36 - 10)i}{2^2 + 3^2} \\[6pt]
            &= 3 + 2i.
  \end{align*}

  The choice of $\lambda$ forced by the equation above indeed satisfies the second required equation $\lambda(5 + 4i) = 7 + 22i$. However, with this choice of $\lambda$ we have $\lambda(-6 + 7i) = -32 + 9i$, which shows that the third required equation is not satisfied.

  Thus no choice of $\lambda \in \C$ satisfies all three required equations.
\end{solution}
