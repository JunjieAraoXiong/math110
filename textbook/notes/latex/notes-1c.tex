%==============================================================================
% MATH 110 - Section 1C Notes: Subspaces
% Linear Algebra Done Right, 4th ed. - Sheldon Axler
%==============================================================================

\documentclass[10pt, twocolumn]{article}
\usepackage[margin=0.75in, columnsep=0.3in]{geometry}

%------------------------------------------------------------------------------
% PACKAGES
%------------------------------------------------------------------------------
\usepackage{amsmath, amssymb, amsthm}
\usepackage{enumitem}
\usepackage{fancyhdr}
\usepackage{titlesec}
\usepackage{tcolorbox}
\usepackage{booktabs}
\usepackage{tikz}
\usetikzlibrary{matrix, arrows.meta, positioning, calc, cd}
\usepackage{graphicx}

%------------------------------------------------------------------------------
% SPACING
%------------------------------------------------------------------------------
\linespread{1.08}
\setlength{\parskip}{0.4ex plus 0.2ex minus 0.1ex}

%------------------------------------------------------------------------------
% BOX STYLES
%------------------------------------------------------------------------------
\tcbset{
    boxrule=0.8pt,
    colback=white,
    colframe=black,
    arc=0pt,
    boxsep=3pt,
    left=4pt, right=4pt, top=4pt, bottom=4pt
}

\newtcolorbox{result}{
    boxrule=0pt,
    colback=black!5,
    colframe=white,
    arc=0pt,
    boxsep=2pt,
    left=4pt, right=4pt, top=4pt, bottom=4pt
}

%------------------------------------------------------------------------------
% SECTION FORMATTING
%------------------------------------------------------------------------------
\titleformat{\section}{\large\bfseries}{\thesection.}{0.5em}{}
\titleformat{\subsection}{\normalsize\bfseries}{\thesubsection}{0.5em}{}
\titlespacing*{\section}{0pt}{1.5ex}{1ex}
\titlespacing*{\subsection}{0pt}{1ex}{0.5ex}

%------------------------------------------------------------------------------
% LIST FORMATTING
%------------------------------------------------------------------------------
\setlist{itemsep=1pt, topsep=3pt, parsep=1pt, leftmargin=1.5em}

%------------------------------------------------------------------------------
% HEADER/FOOTER
%------------------------------------------------------------------------------
\pagestyle{fancy}
\fancyhf{}
\fancyhead[L]{\small MATH 110}
\fancyhead[R]{\small Section 1C}
\fancyfoot[C]{\small\thepage}
\renewcommand{\headrulewidth}{0.4pt}

%------------------------------------------------------------------------------
% THEOREM ENVIRONMENTS
%------------------------------------------------------------------------------
\theoremstyle{definition}
\newtheorem{property}{Property}
\newtheorem{definition}{Definition}
\newtheorem{example}{Example}

\theoremstyle{plain}
\newtheorem{theorem}{Theorem}
\newtheorem{lemma}{Lemma}
\newtheorem{proposition}{Proposition}
\newtheorem{corollary}{Corollary}

%------------------------------------------------------------------------------
% CUSTOM COMMANDS
%------------------------------------------------------------------------------
% Number fields
\newcommand{\R}{\mathbb{R}}
\newcommand{\C}{\mathbb{C}}
\newcommand{\F}{\mathbb{F}}
\newcommand{\Z}{\mathbb{Z}}
\newcommand{\Q}{\mathbb{Q}}

% Matrix/space operations
\DeclareMathOperator{\rank}{rank}
\DeclareMathOperator{\nullity}{nullity}
\DeclareMathOperator{\tr}{tr}
\DeclareMathOperator{\spn}{span}
\DeclareMathOperator{\col}{col}
\DeclareMathOperator{\row}{row}
\DeclareMathOperator{\nul}{null}
\DeclareMathOperator{\range}{range}
\DeclareMathOperator{\im}{im}

% Inner products and norms
\newcommand{\inner}[2]{\langle #1, #2 \rangle}
\newcommand{\norm}[1]{\| #1 \|}
\newcommand{\proj}{\operatorname{proj}}

% Vectors and matrices
\newcommand{\vect}[1]{\mathbf{#1}}
\newcommand{\mat}[1]{\mathbf{#1}}

%==============================================================================
\begin{document}
%==============================================================================

\noindent
\begin{minipage}{\linewidth}
    \centering
    \textbf{\Large Section 1C: Subspaces} \\[0.5em]
    \hrule
\end{minipage}
\vspace{1em}

%==============================================================================
\section{Introduction to Subspaces}
%==============================================================================

Often we want to find vector spaces that ``live inside'' a larger vector space. For example, lines and planes through the origin in $\R^3$ turn out to be vector spaces in their own right. The key idea is that a subset can inherit the vector space structure from its parent.

\begin{tcolorbox}
\textbf{1.33 Definition: Subspace}

A subset $U$ of $V$ is called a \textbf{subspace} of $V$ if $U$ is also a vector space (using the same addition and scalar multiplication as on $V$).
\end{tcolorbox}

\textbf{Key point:} A subspace uses the \textit{same} operations as the parent space. We don't invent new addition or scalar multiplication---we just restrict to a subset.

\textbf{Note:} The additive identity of a subspace must be the same as the additive identity of the larger space. If $0_U$ is the additive identity of $U$ and $0$ is the additive identity of $V$, then for any $v \in U$:
\[
    0_U = 0_U + 0 = 0 + 0_U = 0
\]
So subspaces always contain the zero vector of $V$.

%==============================================================================
\section{The Subspace Test}
%==============================================================================

Checking all 8 vector space axioms would be tedious. Fortunately, most axioms are inherited automatically. The following result gives us a simpler test.

\begin{tcolorbox}
\textbf{1.34 Conditions for a Subspace}

A subset $U$ of $V$ is a subspace of $V$ if and only if $U$ satisfies the following three conditions:

\textbf{additive identity:}
\[
    0 \in U
\]

\textbf{closed under addition:}
\[
    u, w \in U \implies u + w \in U
\]

\textbf{closed under scalar multiplication:}
\[
    a \in \F \text{ and } u \in U \implies au \in U
\]
\end{tcolorbox}

\textbf{Proof sketch:}

$(\Rightarrow)$ If $U$ is a subspace, it's a vector space, so it has an additive identity. Since the additive identity is unique in $V$ (by 1.26), it must be $0$. Closure under addition and scalar multiplication follow because these are operations on $U$.

$(\Leftarrow)$ Suppose $U$ satisfies the three conditions. We verify $U$ is a vector space:
\begin{itemize}
    \item The three conditions give us $0 \in U$, closure under $+$, and closure under scalar multiplication.
    \item For $u \in U$: $-u = (-1)u \in U$ by closure under scalar multiplication. So additive inverses exist in $U$.
    \item Commutativity, associativity, distributivity, and the multiplicative identity hold because they hold in $V$ and $U \subseteq V$.
\end{itemize}
\hfill $\square$

\begin{result}
\textbf{Alternative condition:} Instead of checking $0 \in U$, you can check that $U$ is \textbf{nonempty}. Here's why:

If $U \neq \emptyset$, pick any $u \in U$. By closure under scalar multiplication:
\[
    0 \cdot u = 0 \in U
\]

So ``$U$ is nonempty'' $\Leftrightarrow$ ``$0 \in U$'' when the other conditions hold.
\end{result}

%==============================================================================
\section{Examples of Subspaces}
%==============================================================================

\begin{tcolorbox}[colframe=black!50]
\textbf{1.35(a) Example: $\{(x_1, x_2, x_3, x_4) \in \F^4 : x_3 = 5x_4 + b\}$}

For which values of $b \in \F$ is this set a subspace of $\F^4$?

\textbf{Answer:} Only when $b = 0$.

\textbf{If $b = 0$:} The set is $\{(x_1, x_2, x_3, x_4) : x_3 = 5x_4\}$.
\begin{itemize}
    \item $0 = (0,0,0,0)$ satisfies $0 = 5 \cdot 0$. \checkmark
    \item Closed under $+$: If $x_3 = 5x_4$ and $y_3 = 5y_4$, then $x_3 + y_3 = 5(x_4 + y_4)$. \checkmark
    \item Closed under scalar mult: If $x_3 = 5x_4$, then $\lambda x_3 = 5(\lambda x_4)$. \checkmark
\end{itemize}

\textbf{If $b \neq 0$:} The zero vector $(0,0,0,0)$ does not satisfy $0 = 5 \cdot 0 + b = b$. So $0 \notin U$, and $U$ is not a subspace.
\end{tcolorbox}

\textbf{Lesson:} Subspaces must pass through the origin. A constraint like $x_3 = 5x_4 + b$ with $b \neq 0$ defines an ``affine subspace'' (a shifted subspace), not a true subspace.

\begin{tcolorbox}[colframe=black!50]
\textbf{1.35(b) Example: Continuous Functions}

The set of continuous real-valued functions on $[0,1]$ is a subspace of $\R^{[0,1]}$ (the space of all functions from $[0,1]$ to $\R$).

\textbf{Verification:}
\begin{itemize}
    \item The zero function $0(x) = 0$ is continuous. \checkmark
    \item Sum of continuous functions is continuous. \checkmark
    \item Scalar multiple of a continuous function is continuous. \checkmark
\end{itemize}
\end{tcolorbox}

\begin{tcolorbox}[colframe=black!50]
\textbf{1.35(c) Example: Differentiable Functions}

The set of differentiable real-valued functions on $\R$ is a subspace of $\R^\R$.

\textbf{Verification:}
\begin{itemize}
    \item The zero function is differentiable (with derivative $0$). \checkmark
    \item Sum of differentiable functions is differentiable. \checkmark
    \item Scalar multiple of a differentiable function is differentiable. \checkmark
\end{itemize}

\textbf{Note:} We also have a chain of subspaces:
\[
    \{\text{polynomials}\} \subseteq \{C^\infty\} \subseteq \{C^1\} \subseteq \{C^0\} \subseteq \R^\R
\]
\end{tcolorbox}

\begin{tcolorbox}[colframe=black!50]
\textbf{1.35(d) Example: $\{f \in \R^\R : f \text{ differentiable and } f'(2) = b\}$}

For which values of $b \in \R$ is this a subspace?

\textbf{Answer:} Only when $b = 0$.

\textbf{If $b = 0$:} This is the set of differentiable functions whose derivative vanishes at $2$.
\begin{itemize}
    \item The zero function has $0'(2) = 0$. \checkmark
    \item If $f'(2) = 0$ and $g'(2) = 0$, then $(f+g)'(2) = 0$. \checkmark
    \item If $f'(2) = 0$, then $(\lambda f)'(2) = \lambda \cdot 0 = 0$. \checkmark
\end{itemize}

\textbf{If $b \neq 0$:} The zero function has $0'(2) = 0 \neq b$, so $0 \notin U$.
\end{tcolorbox}

\begin{tcolorbox}[colframe=black!50]
\textbf{1.35(e) Example: Sequences Converging to 0}

The set
\[
    \{(a_1, a_2, \ldots) \in \C^\infty : \lim_{n \to \infty} a_n = 0\}
\]
is a subspace of $\C^\infty$.

\textbf{Verification:}
\begin{itemize}
    \item The zero sequence $(0, 0, \ldots)$ has limit $0$. \checkmark
    \item If $\lim a_n = 0$ and $\lim b_n = 0$, then $\lim(a_n + b_n) = 0$. \checkmark
    \item If $\lim a_n = 0$, then $\lim(\lambda a_n) = \lambda \cdot 0 = 0$. \checkmark
\end{itemize}
\end{tcolorbox}

\begin{result}
\textbf{Extreme subspaces:} Every vector space $V$ has two ``trivial'' subspaces:
\begin{itemize}
    \item The \textbf{smallest subspace}: $\{0\}$ (just the zero vector)
    \item The \textbf{largest subspace}: $V$ itself
\end{itemize}

\textbf{Why the empty set is not a subspace:} $\emptyset$ fails the condition $0 \in U$. Every subspace must contain the zero vector.
\end{result}

%==============================================================================
\section{Sums of Subspaces}
%==============================================================================

Given subspaces $U_1, \ldots, U_m$ of $V$, we want to build new subspaces from them. The natural candidate---the union---usually fails.

\textbf{Why not unions?} If $U$ and $W$ are subspaces of $V$, then $U \cup W$ is \textit{usually not} a subspace. For example, in $\R^2$:
\begin{itemize}
    \item Let $U = \{(x, 0) : x \in \R\}$ (the $x$-axis)
    \item Let $W = \{(0, y) : y \in \R\}$ (the $y$-axis)
    \item Then $(1, 0) \in U$ and $(0, 1) \in W$
    \item But $(1, 0) + (0, 1) = (1, 1) \notin U \cup W$
\end{itemize}

The union is not closed under addition! We need something else.

\begin{tcolorbox}
\textbf{1.36 Definition: Sum of Subspaces}

Suppose $U_1, \ldots, U_m$ are subspaces of $V$. The \textbf{sum} of $U_1, \ldots, U_m$, denoted $U_1 + \cdots + U_m$, is the set of all possible sums of elements of $U_1, \ldots, U_m$:
\[
    U_1 + \cdots + U_m = \{u_1 + \cdots + u_m : u_j \in U_j\}
\]
\end{tcolorbox}

\textbf{Key insight:} The sum $U_1 + \cdots + U_m$ is the \textit{smallest} subspace of $V$ containing all of $U_1, \ldots, U_m$.

\textbf{Verification that the sum is a subspace:}
\begin{itemize}
    \item $0 = 0 + \cdots + 0 \in U_1 + \cdots + U_m$ \checkmark
    \item Closed under addition: $(u_1 + \cdots + u_m) + (v_1 + \cdots + v_m) = (u_1 + v_1) + \cdots + (u_m + v_m)$, and each $u_j + v_j \in U_j$. \checkmark
    \item Closed under scalar mult: $\lambda(u_1 + \cdots + u_m) = \lambda u_1 + \cdots + \lambda u_m$, and each $\lambda u_j \in U_j$. \checkmark
\end{itemize}

\begin{tcolorbox}[colframe=black!50]
\textbf{1.37 Example: Sum of Subspaces of $\F^3$}

Let $U = \{(x, 0, 0) \in \F^3 : x \in \F\}$ (the $x$-axis).

Let $W = \{(0, y, 0) \in \F^3 : y \in \F\}$ (the $y$-axis).

Then:
\[
    U + W = \{(x, y, 0) \in \F^3 : x, y \in \F\}
\]

This is the $xy$-plane! A general element of $U + W$ is:
\[
    (x, 0, 0) + (0, y, 0) = (x, y, 0)
\]
\end{tcolorbox}

\begin{tcolorbox}[colframe=black!50]
\textbf{1.38 Example: Sum of Subspaces of $\F^4$}

Let $U = \{(x, x, y, y) \in \F^4 : x, y \in \F\}$.

Let $W = \{(x, x, x, y) \in \F^4 : x, y \in \F\}$.

\textbf{Claim:} $U + W = \{(x, x, y, z) \in \F^4 : x, y, z \in \F\}$.

\textbf{Proof of $\subseteq$:} Take $u \in U$ and $w \in W$:
\begin{align*}
    u &= (a, a, b, b) \quad \text{for some } a, b \in \F \\
    w &= (c, c, c, d) \quad \text{for some } c, d \in \F \\
    u + w &= (a+c, a+c, b+c, b+d)
\end{align*}
Notice that the first two coordinates are equal. So $u + w$ has the form $(x, x, y, z)$.

\textbf{Proof of $\supseteq$:} \textbf{(1.39)} Given $(x, x, y, z) \in \F^4$, we want to write it as $u + w$ with $u \in U$ and $w \in W$.

Choose:
\begin{align*}
    u &= (x, x, y, y) \in U \\
    w &= (0, 0, 0, z-y) \in W \quad \text{(since } (0, 0, 0, z-y) \text{ has form } (t, t, t, s) \text{ with } t=0)
\end{align*}

Then $u + w = (x, x, y, y) + (0, 0, 0, z-y) = (x, x, y, z)$. \checkmark
\end{tcolorbox}

%==============================================================================
\newpage
\section*{Strategy: How to Check if $U$ is a Subspace}
%==============================================================================

\begin{result}
\textbf{The Subspace Checklist:}

To verify that $U \subseteq V$ is a subspace, check three things:

\begin{enumerate}
    \item \textbf{Zero vector:} Show $0 \in U$.

    (Alternatively, show $U \neq \emptyset$.)

    \item \textbf{Closed under addition:} Take \textit{arbitrary} $u, w \in U$. Show that $u + w \in U$.

    (Use the definition of $U$ to verify the sum satisfies the defining property.)

    \item \textbf{Closed under scalar multiplication:} Take \textit{arbitrary} $a \in \F$ and $u \in U$. Show that $au \in U$.
\end{enumerate}

\textbf{Common mistakes to avoid:}
\begin{itemize}
    \item Don't use specific vectors; use \textit{arbitrary} elements.
    \item Don't forget to check $0 \in U$ (or nonemptiness).
    \item Remember: the operations come from $V$, not something new.
\end{itemize}
\end{result}

\begin{result}
\textbf{Quick tests for NON-subspaces:}

A subset $U \subseteq V$ is \textbf{NOT} a subspace if any of these hold:
\begin{itemize}
    \item $0 \notin U$ (e.g., $\{x : x_1 = 1\}$)
    \item Not closed under $+$ (find $u, w \in U$ with $u + w \notin U$)
    \item Not closed under scalar mult (find $a \in \F$, $u \in U$ with $au \notin U$)
\end{itemize}

\textbf{Rule of thumb:} Constraints of the form ``$= b$'' with $b \neq 0$ usually fail the subspace test because $0$ won't satisfy the constraint.
\end{result}

%==============================================================================
\section*{Key Results Summary}
%==============================================================================

\begin{tcolorbox}
\textbf{Definitions:}
\begin{itemize}
    \item \textbf{Subspace} (1.33): A subset that is itself a vector space with the inherited operations
    \item \textbf{Sum of subspaces} (1.36): $U_1 + \cdots + U_m = \{u_1 + \cdots + u_m : u_j \in U_j\}$
\end{itemize}

\textbf{The Subspace Test} (1.34):

$U$ is a subspace of $V$ $\Leftrightarrow$
\begin{enumerate}
    \item $0 \in U$
    \item $u, w \in U \Rightarrow u + w \in U$
    \item $a \in \F, u \in U \Rightarrow au \in U$
\end{enumerate}

\textbf{Key Facts:}
\begin{itemize}
    \item Every subspace contains $0$
    \item $\{0\}$ and $V$ are always subspaces of $V$
    \item The empty set is never a subspace
    \item Sums of subspaces are subspaces
    \item Unions of subspaces are usually NOT subspaces
\end{itemize}
\end{tcolorbox}

\begin{tcolorbox}
\textbf{Common Problem Types:}

\begin{description}[style=nextline, leftmargin=1em, labelindent=0pt]
    \item[Determine if $U$ is a subspace]
    Use the three-condition test. Check $0 \in U$, closure under $+$, closure under scalar mult.

    \item[For which $b$ is $U$ a subspace?]
    Usually $b = 0$. Check whether $0$ satisfies the defining condition.

    \item[Describe $U + W$]
    Write a general element as $u + w$ where $u \in U$, $w \in W$. Simplify to find the pattern.

    \item[Prove $U + W = $ some set $S$]
    Show $U + W \subseteq S$ (every sum has the right form) and $S \subseteq U + W$ (every element of $S$ can be written as a sum).
\end{description}
\end{tcolorbox}

\end{document}
