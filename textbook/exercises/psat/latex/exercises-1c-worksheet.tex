%==============================================================================
% EXERCISES 1C WORKSHEET
% Linear Algebra Done Right (4th ed.) - Sheldon Axler
%==============================================================================

\documentclass[11pt]{article}
\usepackage[margin=1in]{geometry}

% Required packages
\usepackage{amsmath, amssymb, amsthm}
\usepackage{enumitem}
\usepackage{fancyhdr}
\usepackage{tikz}
\usepackage{xcolor}

% Custom commands (from style guide)
\newcommand{\R}{\mathbb{R}}
\newcommand{\C}{\mathbb{C}}
\newcommand{\F}{\mathbb{F}}
\newcommand{\Z}{\mathbb{Z}}

% Header/footer
\pagestyle{fancy}
\fancyhf{}
\fancyhead[L]{\small MATH 110}
\fancyhead[R]{\small Exercises 1C}
\fancyfoot[C]{\small\thepage}
\renewcommand{\headrulewidth}{0.4pt}

% Exercise counter and command
\newcounter{exercise}
\newcommand{\exercise}{\refstepcounter{exercise}\noindent\textbf{Exercise \theexercise.} }

% Ruled solution space with gray separator
\newcommand{\ruledspace}[1]{%
  \par\vspace{0.3em}%
  \noindent\textcolor{gray!30}{\hrule}%
  \vspace{#1}%
}

% Answer line for computation problems
\newcommand{\answerline}{%
  \vspace{0.5em}%
  \noindent\textbf{Answer:} \hrulefill%
  \vspace{1em}%
}

\begin{document}

\noindent
\begin{minipage}{\linewidth}
    \centering
    \textbf{\Large Exercises 1C: Subspaces} \\[0.5em]
    \textit{Linear Algebra Done Right, 4th ed.}\\[0.3em]
    \hrule
\end{minipage}
\vspace{1.5em}

%==============================================================================
% INTERSECTION AND UNION OF SUBSPACES (11-13)
%==============================================================================

\exercise
Prove that the intersection of every collection of subspaces of $V$ is a subspace of $V$.
\ruledspace{7cm}

%------------------------------------------------------------------------------

\exercise
Prove that the union of two subspaces of $V$ is a subspace of $V$ if and only if one of the subspaces is contained in the other.
\ruledspace{8cm}

\newpage

%------------------------------------------------------------------------------

\exercise
Prove that the union of three subspaces of $V$ is a subspace of $V$ if and only if one of the subspaces contains the other two.

\textit{Hint:} This exercise is more complicated than the previous exercise; there is no quick argument. First prove this in the special case where $\F$ has only two elements, and then prove it when $\F$ has more than two elements.
\ruledspace{10cm}

\newpage

%==============================================================================
% SUMS OF SUBSPACES (14-18)
%==============================================================================

\exercise
Suppose $U$ is the subspace of $\F^3$ defined by
\[
    U = \{(x, -x, 2x) \in \F^3 : x \in \F\}.
\]
Also suppose $W$ is the subspace of $\F^3$ defined by
\[
    W = \{(x, x, 2x) \in \F^3 : x \in \F\}.
\]
Describe $U + W$ both symbolically and non-symbolically.
\ruledspace{6cm}
\answerline

%------------------------------------------------------------------------------

\exercise
Suppose $U$ is a subspace of $V$. What is $U + U$?
\ruledspace{4cm}
\answerline

\newpage

%------------------------------------------------------------------------------

\exercise
Is the operation of addition on the subspaces of $V$ commutative? In other words, if $U$ and $W$ are subspaces of $V$, is $U + W = W + U$?
\ruledspace{5cm}

%------------------------------------------------------------------------------

\exercise
Is the operation of addition on the subspaces of $V$ associative? In other words, if $V_1$, $V_2$, $V_3$ are subspaces of $V$, is
\[
    (V_1 + V_2) + V_3 = V_1 + (V_2 + V_3)?
\]
\ruledspace{6cm}

\newpage

%------------------------------------------------------------------------------

\exercise
Does the operation of addition on the subspaces of $V$ have an additive identity? Which subspaces have additive inverses?
\ruledspace{6cm}

%==============================================================================
% PROVE OR GIVE A COUNTEREXAMPLE (19, 23)
%==============================================================================

\exercise
Suppose $V_1$, $V_2$, and $U$ are subspaces of $V$. Prove or give a counterexample: If $V_1 + U = V_2 + U$, then $V_1 = V_2$.
\ruledspace{7cm}

\newpage

%------------------------------------------------------------------------------

\exercise
Suppose $V_1$, $V_2$, and $U$ are subspaces of $V$. Prove or give a counterexample: If $V = V_1 \oplus U$ and $V = V_2 \oplus U$, then $V_1 = V_2$.

\textit{Hint:} Think about subspaces of $\F^2$.
\ruledspace{7cm}

%==============================================================================
% DIRECT SUM PROBLEMS (20-22)
%==============================================================================

\exercise
Suppose
\[
    U = \{(x, x, y, y) \in \F^4 : x, y \in \F\}.
\]
Find a subspace $W$ of $\F^4$ such that $\F^4 = U \oplus W$.
\ruledspace{6cm}
\answerline

\newpage

%------------------------------------------------------------------------------

\exercise
Suppose
\[
    U = \{(x, y, x+y, x-y, 2x) \in \F^5 : x, y \in \F\}.
\]
Find a subspace $W$ of $\F^5$ such that $\F^5 = U \oplus W$.
\ruledspace{7cm}
\answerline

%------------------------------------------------------------------------------

\exercise
Suppose
\[
    U = \{(x, y, x+y, x-y, 2x) \in \F^5 : x, y \in \F\}.
\]
Find three nonzero subspaces $W_1$, $W_2$, $W_3$ of $\F^5$, none of which equals $\{0\}$, such that
\[
    \F^5 = U \oplus W_1 \oplus W_2 \oplus W_3.
\]
\ruledspace{7cm}

\newpage

%==============================================================================
% EVEN AND ODD FUNCTIONS (24)
%==============================================================================

\exercise
A function $f: \R \to \R$ is called \textbf{even} if
\[
    f(-x) = f(x) \quad \text{for all } x \in \R.
\]
A function $f: \R \to \R$ is called \textbf{odd} if
\[
    f(-x) = -f(x) \quad \text{for all } x \in \R.
\]
Let $V_e$ denote the set of real-valued even functions on $\R$, and let $V_o$ denote the set of real-valued odd functions on $\R$. Show that
\[
    \R^\R = V_e \oplus V_o.
\]
\ruledspace{10cm}

\vfill

\end{document}
