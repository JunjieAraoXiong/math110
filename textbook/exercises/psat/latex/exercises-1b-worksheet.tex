%==============================================================================
% EXERCISES 1B WORKSHEET
% Linear Algebra Done Right (4th ed.) - Sheldon Axler
%==============================================================================

\documentclass[11pt]{article}
\usepackage[margin=1in]{geometry}

% Required packages
\usepackage{amsmath, amssymb, amsthm}
\usepackage{enumitem}
\usepackage{fancyhdr}
\usepackage{tikz}
\usepackage{xcolor}

% Custom commands (from style guide)
\newcommand{\R}{\mathbb{R}}
\newcommand{\C}{\mathbb{C}}
\newcommand{\F}{\mathbb{F}}
\newcommand{\Z}{\mathbb{Z}}

% Header/footer
\pagestyle{fancy}
\fancyhf{}
\fancyhead[L]{\small MATH 110}
\fancyhead[R]{\small Exercises 1B}
\fancyfoot[C]{\small\thepage}
\renewcommand{\headrulewidth}{0.4pt}

% Exercise counter and command
\newcounter{exercise}
\newcommand{\exercise}{\refstepcounter{exercise}\noindent\textbf{Exercise \theexercise.} }

% Ruled solution space with gray separator
\newcommand{\ruledspace}[1]{%
  \par\vspace{0.3em}%
  \noindent\textcolor{gray!30}{\hrule}%
  \vspace{#1}%
}

% Answer line for computation problems
\newcommand{\answerline}{%
  \vspace{0.5em}%
  \noindent\textbf{Answer:} \hrulefill%
  \vspace{1em}%
}

\begin{document}

\noindent
\begin{minipage}{\linewidth}
    \centering
    \textbf{\Large Exercises 1B: Definition of Vector Space} \\[0.5em]
    \textit{Linear Algebra Done Right, 4th ed.}\\[0.3em]
    \hrule
\end{minipage}
\vspace{1.5em}

%==============================================================================
% BASIC VECTOR SPACE PROOFS (1-4)
%==============================================================================

\exercise
Prove that $-(-v) = v$ for every $v \in V$.
\ruledspace{6cm}

%------------------------------------------------------------------------------

\exercise
Suppose $a \in \F$, $v \in V$, and $av = 0$. Prove that $a = 0$ or $v = 0$.
\ruledspace{7cm}

\newpage

%------------------------------------------------------------------------------

\exercise
Suppose $v, w \in V$. Explain why there exists a unique $x \in V$ such that $v + 3x = w$.
\ruledspace{7cm}

%------------------------------------------------------------------------------

\exercise
The empty set is not a vector space. The empty set fails to satisfy only one of the requirements listed in the definition of a vector space (1.20). Which one?
\ruledspace{5cm}
\answerline

\newpage

%==============================================================================
% ALTERNATIVE AXIOMS AND EXTENDED REALS (5-6)
%==============================================================================

\exercise
Show that in the definition of a vector space (1.20), the additive inverse condition can be replaced with the condition that
\[
    0v = 0 \quad \text{for all } v \in V.
\]
The $0$ on the left side is the number $0$, and the $0$ on the right side is the additive identity of $V$.

\textit{Hint:} The point here is that if the expression ``for every $v \in V$, there exists $w \in V$ such that $v + w = 0$'' in the definition of vector space is replaced with ``$0v = 0$ for all $v \in V$,'' then the other conditions in the definition of vector space imply that for each $v \in V$, there exists an additive inverse of $v$.
\ruledspace{8cm}

\newpage

%------------------------------------------------------------------------------

\exercise
Let $\infty$ and $-\infty$ denote two distinct objects, neither of which is in $\R$. Define an addition and scalar multiplication on $\R \cup \{\infty, -\infty\}$ as follows: the usual operations on $\R$ are unchanged; for $t \in \R$ define
\begin{align*}
    t + \infty &= \infty + t = \infty, \\
    t + (-\infty) &= (-\infty) + t = -\infty, \\
    \infty + \infty &= \infty, \\
    (-\infty) + (-\infty) &= -\infty, \\
    \infty + (-\infty) &= 0;
\end{align*}
for $t \in \R$ with $t > 0$ define
\begin{align*}
    t \cdot \infty &= \infty, & t \cdot (-\infty) &= -\infty;
\end{align*}
for $t \in \R$ with $t < 0$ define
\begin{align*}
    t \cdot \infty &= -\infty, & t \cdot (-\infty) &= \infty;
\end{align*}
and define
\begin{align*}
    0 \cdot \infty &= 0, & 0 \cdot (-\infty) &= 0.
\end{align*}
Is $\R \cup \{\infty, -\infty\}$ a vector space over $\R$? Explain.
\ruledspace{6cm}

\newpage

%==============================================================================
% FUNCTION SPACES AND COMPLEXIFICATION (7-8)
%==============================================================================

\exercise
Suppose $S$ is a nonempty set and $V$ is a vector space. Let $V^S$ denote the set of functions from $S$ to $V$. Define a natural addition and scalar multiplication on $V^S$, and show that $V^S$ is a vector space with these definitions.
\ruledspace{10cm}

\newpage

%------------------------------------------------------------------------------

\exercise
Suppose $V$ is a real vector space. The \textbf{complexification} of $V$, denoted $V_\C$, equals $V \times V$. An element of $V_\C$ is an ordered pair $(u, v)$, where $u, v \in V$, but we will write this as $u + iv$.

Define addition on $V_\C$ by
\[
    (u_1 + iv_1) + (u_2 + iv_2) = (u_1 + u_2) + i(v_1 + v_2)
\]
for $u_1, v_1, u_2, v_2 \in V$.

Define complex scalar multiplication on $V_\C$ by
\[
    (a + bi)(u + iv) = (au - bv) + i(av + bu)
\]
for $a, b \in \R$ and $u, v \in V$.

Prove that with the definitions of addition and scalar multiplication as above, $V_\C$ is a complex vector space.
\ruledspace{10cm}

\vfill

\end{document}
