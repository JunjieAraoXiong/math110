%==============================================================================
% EXERCISES 1A SOLUTIONS
% Linear Algebra Done Right (4th ed.) - Sheldon Axler
%==============================================================================

\documentclass[11pt]{article}
\usepackage[margin=1in]{geometry}

% Required packages
\usepackage{amsmath, amssymb, amsthm}
\usepackage{enumitem}
\usepackage{fancyhdr}
\usepackage{tikz}
\usepackage{xcolor}
\usepackage{tcolorbox}
\usepackage{titlesec}
\usepackage{booktabs}
\usetikzlibrary{matrix, arrows.meta, positioning, calc, cd}

% Improved spacing
\linespread{1.15}
\setlength{\parskip}{0.5ex plus 0.2ex minus 0.1ex}

% Custom commands (from style guide)
\newcommand{\R}{\mathbb{R}}
\newcommand{\C}{\mathbb{C}}
\newcommand{\F}{\mathbb{F}}
\newcommand{\Z}{\mathbb{Z}}

% Custom operators
\DeclareMathOperator{\rank}{rank}
\DeclareMathOperator{\nullity}{nullity}
\DeclareMathOperator{\spn}{span}

% Box environments (from style guide)
\tcbset{
    boxrule=0.8pt,
    colback=white,
    colframe=black,
    arc=0pt,
    boxsep=3pt,
    left=4pt, right=4pt, top=4pt, bottom=4pt
}

\newtcolorbox{solutionbox}{
    boxrule=0.5pt,
    colback=black!3,
    colframe=black!40,
    arc=0pt,
    boxsep=4pt,
    left=6pt, right=6pt, top=6pt, bottom=6pt
}

\newtcolorbox{result}{
    boxrule=0.8pt,
    colback=white,
    colframe=black,
    arc=0pt,
    boxsep=3pt,
    left=4pt, right=4pt, top=4pt, bottom=4pt
}

\newtcolorbox{hintbox}{
    boxrule=0.5pt,
    colback=yellow!5,
    colframe=black!30,
    arc=0pt,
    boxsep=3pt,
    left=4pt, right=4pt, top=4pt, bottom=4pt
}

% Header/footer
\pagestyle{fancy}
\fancyhf{}
\fancyhead[L]{\small MATH 110}
\fancyhead[R]{\small Exercises 1A Solutions}
\fancyfoot[C]{\small\thepage}
\renewcommand{\headrulewidth}{0.4pt}

% Exercise counter and command
\newcounter{exercise}
\newcommand{\exercise}{\refstepcounter{exercise}\noindent\textbf{Exercise \theexercise.} }

% Solution environment
\newenvironment{solution}{\begin{proof}[Solution]}{\end{proof}}

\begin{document}

\noindent
\begin{minipage}{\linewidth}
    \centering
    \textbf{\Large Exercises 1A Solutions: $\R^n$ and $\C^n$} \\[0.5em]
    \textit{Linear Algebra Done Right, 4th ed.}\\[0.3em]
    \hrule
\end{minipage}
\vspace{1.5em}

%==============================================================================
% COMPLEX NUMBER EXERCISES (1-6)
%==============================================================================

\exercise
Suppose $a$ and $b$ are real numbers, not both $0$. Find real numbers $c$ and $d$ such that
\[
    \frac{1}{a + bi} = c + di.
\]

\begin{solution}
Multiply the numerator and denominator by the complex conjugate of the denominator:
\[
    \frac{1}{a + bi} = \frac{1}{a + bi} \cdot \frac{a - bi}{a - bi} = \frac{a - bi}{a^2 + b^2}.
\]
Since $a$ and $b$ are not both $0$, we have $a^2 + b^2 \neq 0$. Separating into real and imaginary parts:
\[
    \frac{1}{a + bi} = \frac{a}{a^2 + b^2} - \frac{b}{a^2 + b^2}i.
\]
Therefore:
\[
    \boxed{c = \frac{a}{a^2 + b^2}, \quad d = \frac{-b}{a^2 + b^2}}
\]
\end{solution}

%------------------------------------------------------------------------------

\exercise
Show that
\[
    \frac{-1 + \sqrt{3}\,i}{2}
\]
is a cube root of $1$ (meaning that its cube equals $1$).

\begin{solutionbox}
\textbf{Solution:} Let $\omega = \frac{-1 + \sqrt{3}\,i}{2}$. We compute $\omega^3$ step by step.

First, compute $\omega^2$:
\begin{align*}
    \omega^2 &= \left(\frac{-1 + \sqrt{3}\,i}{2}\right)^2 = \frac{(-1 + \sqrt{3}\,i)^2}{4} \\[6pt]
    &= \frac{1 - 2\sqrt{3}\,i + 3i^2}{4} = \frac{1 - 2\sqrt{3}\,i - 3}{4} \\[6pt]
    &= \frac{-2 - 2\sqrt{3}\,i}{4} = \frac{-1 - \sqrt{3}\,i}{2}.
\end{align*}

Now compute $\omega^3 = \omega^2 \cdot \omega$:
\begin{align*}
    \omega^3 &= \frac{-1 - \sqrt{3}\,i}{2} \cdot \frac{-1 + \sqrt{3}\,i}{2} \\[6pt]
    &= \frac{(-1 - \sqrt{3}\,i)(-1 + \sqrt{3}\,i)}{4} \\[6pt]
    &= \frac{(-1)^2 - (\sqrt{3}\,i)^2}{4} = \frac{1 - 3i^2}{4} = \frac{1 + 3}{4} = 1.
\end{align*}

Therefore $\omega^3 = 1$, so $\omega$ is a cube root of $1$. \hfill $\square$
\end{solutionbox}

%------------------------------------------------------------------------------

\exercise
Find two distinct square roots of $i$.

\begin{solutionbox}
\textbf{Solution:} We seek $z = a + bi$ such that $z^2 = i$, where $a, b \in \R$.

Expanding $z^2$:
\[
    z^2 = (a + bi)^2 = a^2 + 2abi + b^2i^2 = (a^2 - b^2) + 2abi.
\]

Setting $z^2 = i = 0 + 1 \cdot i$, we equate real and imaginary parts:
\begin{align*}
    a^2 - b^2 &= 0 \\
    2ab &= 1
\end{align*}

From the first equation, $a^2 = b^2$, so $a = \pm b$.

If $a = b$: From $2ab = 1$, we get $2a^2 = 1$, so $a = \pm\frac{1}{\sqrt{2}}$.
This gives $a = b = \frac{1}{\sqrt{2}}$ or $a = b = -\frac{1}{\sqrt{2}}$.

If $a = -b$: From $2ab = 1$, we get $-2a^2 = 1$, which has no real solutions.

Therefore the two square roots of $i$ are:
\[
    \boxed{z_1 = \frac{1}{\sqrt{2}} + \frac{1}{\sqrt{2}}i = \frac{\sqrt{2}}{2}(1 + i), \quad z_2 = -\frac{1}{\sqrt{2}} - \frac{1}{\sqrt{2}}i = -\frac{\sqrt{2}}{2}(1 + i)}
\]
\end{solutionbox}

\begin{hintbox}
\textbf{Intuition:} The phrase ``two distinct square roots of $i$'' simply means: find two \emph{different} complex numbers whose square equals $i$.

\textit{Why do two exist?} If $z^2 = i$, then $(-z)^2 = z^2 = i$ as well. So square roots come in $\pm$ pairs. Unless $z = 0$, these two are different.

\textit{Why say ``distinct''?} To emphasize we want two different answers, not the same number listed twice.

\textit{Why does ``no real solutions'' matter for complex numbers?} The solution writes $z = a + bi$ where $a, b \in \R$. This is crucial: even though $z$ is complex, the components $a$ and $b$ are \emph{required to be real} by the definition of complex number notation. When $a = -b$, we get $-2a^2 = 1$, which has no solution in $\R$. Since $a$ must be real, this case is impossible.

\textit{Geometric view:} Squaring a complex number doubles its angle. The angles $\frac{\pi}{4}$ and $\frac{5\pi}{4}$ both double to $\frac{\pi}{2}$ (the angle of $i$). This is why every nonzero complex number has exactly two square roots.
\end{hintbox}

\newpage

%------------------------------------------------------------------------------

\exercise
Show that $\alpha + \beta = \beta + \alpha$ for all $\alpha, \beta \in \C$.

\begin{solution}
Let $\alpha = a + bi$ and $\beta = c + di$ where $a, b, c, d \in \R$.

Computing $\alpha + \beta$:
\[
    \alpha + \beta = (a + bi) + (c + di) = (a + c) + (b + d)i.
\]

Computing $\beta + \alpha$:
\[
    \beta + \alpha = (c + di) + (a + bi) = (c + a) + (d + b)i.
\]

Since addition of real numbers is commutative ($a + c = c + a$ and $b + d = d + b$):
\[
    \alpha + \beta = (a + c) + (b + d)i = (c + a) + (d + b)i = \beta + \alpha.
\]

Therefore $\alpha + \beta = \beta + \alpha$ for all $\alpha, \beta \in \C$. \hfill $\square$
\end{solution}

%------------------------------------------------------------------------------

\exercise
Show that $(\alpha + \beta) + \lambda = \alpha + (\beta + \lambda)$ for all $\alpha, \beta, \lambda \in \C$.

\begin{solution}
Let $\alpha = a + bi$, $\beta = c + di$, and $\lambda = e + fi$ where $a, b, c, d, e, f \in \R$.

Computing the left-hand side:
\begin{align*}
    (\alpha + \beta) + \lambda &= \bigl((a + c) + (b + d)i\bigr) + (e + fi) \\
    &= \bigl((a + c) + e\bigr) + \bigl((b + d) + f\bigr)i.
\end{align*}

Computing the right-hand side:
\begin{align*}
    \alpha + (\beta + \lambda) &= (a + bi) + \bigl((c + e) + (d + f)i\bigr) \\
    &= \bigl(a + (c + e)\bigr) + \bigl(b + (d + f)\bigr)i.
\end{align*}

Since addition of real numbers is associative:
\[
    (a + c) + e = a + (c + e) \quad \text{and} \quad (b + d) + f = b + (d + f).
\]

Therefore $(\alpha + \beta) + \lambda = \alpha + (\beta + \lambda)$. \hfill $\square$
\end{solution}

%------------------------------------------------------------------------------

\exercise
Show that $(\alpha\beta)\lambda = \alpha(\beta\lambda)$ for all $\alpha, \beta, \lambda \in \C$.

\begin{solutionbox}
\textbf{Solution:} Let $\alpha = a + bi$, $\beta = c + di$, and $\lambda = e + fi$ where $a, b, c, d, e, f \in \R$.

First, compute $\alpha\beta$:
\[
    \alpha\beta = (a + bi)(c + di) = (ac - bd) + (ad + bc)i.
\]

Then compute $(\alpha\beta)\lambda$:
\begin{align*}
    (\alpha\beta)\lambda &= \bigl((ac - bd) + (ad + bc)i\bigr)(e + fi) \\
    &= \bigl((ac - bd)e - (ad + bc)f\bigr) + \bigl((ac - bd)f + (ad + bc)e\bigr)i \\
    &= (ace - bde - adf - bcf) + (acf - bdf + ade + bce)i.
\end{align*}

Now compute $\beta\lambda$:
\[
    \beta\lambda = (c + di)(e + fi) = (ce - df) + (cf + de)i.
\]

Then compute $\alpha(\beta\lambda)$:
\begin{align*}
    \alpha(\beta\lambda) &= (a + bi)\bigl((ce - df) + (cf + de)i\bigr) \\
    &= \bigl(a(ce - df) - b(cf + de)\bigr) + \bigl(a(cf + de) + b(ce - df)\bigr)i \\
    &= (ace - adf - bcf - bde) + (acf + ade + bce - bdf)i.
\end{align*}

Comparing terms, we see that LHS $=$ RHS. Therefore $(\alpha\beta)\lambda = \alpha(\beta\lambda)$. \hfill $\square$
\end{solutionbox}

\newpage

%==============================================================================
% MORE COMPLEX NUMBER EXERCISES (7-8)
%==============================================================================

\exercise
Show that for every $\alpha \in \C$, there exists a unique $\beta \in \C$ such that $\alpha + \beta = 0$.

\begin{solution}
Let $\alpha = a + bi$ where $a, b \in \R$.

\textbf{Existence:} Define $\beta = -a + (-b)i = -a - bi$. Then:
\[
    \alpha + \beta = (a + bi) + (-a - bi) = (a - a) + (b - b)i = 0 + 0i = 0.
\]
So such a $\beta$ exists.

\textbf{Uniqueness:} Suppose $\beta_1$ and $\beta_2$ both satisfy $\alpha + \beta_1 = 0$ and $\alpha + \beta_2 = 0$.

Then:
\begin{align*}
    \beta_1 &= \beta_1 + 0 = \beta_1 + (\alpha + \beta_2) \\
    &= (\beta_1 + \alpha) + \beta_2 = (\alpha + \beta_1) + \beta_2 \\
    &= 0 + \beta_2 = \beta_2.
\end{align*}

Therefore $\beta$ is unique. \hfill $\square$
\end{solution}

%------------------------------------------------------------------------------

\exercise
Show that for every $\alpha \in \C$ with $\alpha \neq 0$, there exists a unique $\beta \in \C$ such that $\alpha\beta = 1$.

\begin{solutionbox}
\textbf{Solution:} Let $\alpha = a + bi$ where $a, b \in \R$ and $a$ and $b$ are not both zero.

\textbf{Existence:} Define $\beta = \frac{a}{a^2 + b^2} - \frac{b}{a^2 + b^2}i$.

Note that $a^2 + b^2 \neq 0$ since $\alpha \neq 0$, so $\beta$ is well-defined. Then:
\begin{align*}
    \alpha\beta &= (a + bi)\left(\frac{a}{a^2 + b^2} - \frac{b}{a^2 + b^2}i\right) \\[6pt]
    &= \frac{a^2}{a^2 + b^2} - \frac{ab}{a^2 + b^2}i + \frac{ab}{a^2 + b^2}i - \frac{b^2}{a^2 + b^2}i^2 \\[6pt]
    &= \frac{a^2}{a^2 + b^2} + \frac{b^2}{a^2 + b^2} = \frac{a^2 + b^2}{a^2 + b^2} = 1.
\end{align*}

\textbf{Uniqueness:} Suppose $\beta_1$ and $\beta_2$ both satisfy $\alpha\beta_1 = 1$ and $\alpha\beta_2 = 1$.

Then:
\begin{align*}
    \beta_1 &= \beta_1 \cdot 1 = \beta_1(\alpha\beta_2) = (\beta_1\alpha)\beta_2 = (\alpha\beta_1)\beta_2 = 1 \cdot \beta_2 = \beta_2.
\end{align*}

Therefore $\beta$ is unique. \hfill $\square$
\end{solutionbox}

\newpage

%==============================================================================
% VECTOR EXERCISES (9-10)
%==============================================================================

\exercise
Find $x \in \R^4$ such that
\[
    (4, -3, 1, 7) + 2x = (5, 9, -6, 8).
\]

\begin{solution}
Isolate $2x$ by subtracting $(4, -3, 1, 7)$ from both sides:
\[
    2x = (5, 9, -6, 8) - (4, -3, 1, 7) = (5 - 4, 9 - (-3), -6 - 1, 8 - 7) = (1, 12, -7, 1).
\]

Divide by $2$:
\[
    x = \frac{1}{2}(1, 12, -7, 1) = \left(\frac{1}{2}, 6, -\frac{7}{2}, \frac{1}{2}\right).
\]

Therefore:
\[
    \boxed{x = \left(\frac{1}{2}, 6, -\frac{7}{2}, \frac{1}{2}\right)}
\]
\end{solution}

%------------------------------------------------------------------------------

\exercise
Explain why there does not exist $\lambda \in \C$ such that
\[
    \lambda(2 - 3i, 5 + 4i, -6 + 7i) = (12 - 5i, 7 + 22i, -32 - 9i).
\]

\begin{solution}
If such a $\lambda$ exists, then each component equation must hold simultaneously.

From the first component: $\lambda(2 - 3i) = 12 - 5i$.

Solving for $\lambda$:
\[
    \lambda = \frac{12 - 5i}{2 - 3i} = \frac{(12 - 5i)(2 + 3i)}{(2 - 3i)(2 + 3i)} = \frac{24 + 36i - 10i - 15i^2}{4 + 9} = \frac{24 + 26i + 15}{13} = \frac{39 + 26i}{13} = 3 + 2i.
\]

Now check the second component with $\lambda = 3 + 2i$:
\[
    (3 + 2i)(5 + 4i) = 15 + 12i + 10i + 8i^2 = 15 + 22i - 8 = 7 + 22i. \quad \checkmark
\]

Check the third component with $\lambda = 3 + 2i$:
\[
    (3 + 2i)(-6 + 7i) = -18 + 21i - 12i + 14i^2 = -18 + 9i - 14 = -32 + 9i.
\]

But we need $-32 - 9i$, not $-32 + 9i$.

Since the imaginary parts differ ($9i \neq -9i$), there is no single $\lambda \in \C$ that satisfies all three component equations simultaneously. \hfill $\square$
\end{solution}

\newpage

%==============================================================================
% F^n PROPERTIES (11-15)
%==============================================================================

\exercise
Show that $(x + y) + z = x + (y + z)$ for all $x, y, z \in \F^n$.

\begin{solution}
Let $x = (x_1, \ldots, x_n)$, $y = (y_1, \ldots, y_n)$, and $z = (z_1, \ldots, z_n)$ where $x_j, y_j, z_j \in \F$.

Computing the left-hand side:
\begin{align*}
    (x + y) + z &= (x_1 + y_1, \ldots, x_n + y_n) + (z_1, \ldots, z_n) \\
    &= ((x_1 + y_1) + z_1, \ldots, (x_n + y_n) + z_n).
\end{align*}

Computing the right-hand side:
\begin{align*}
    x + (y + z) &= (x_1, \ldots, x_n) + (y_1 + z_1, \ldots, y_n + z_n) \\
    &= (x_1 + (y_1 + z_1), \ldots, x_n + (y_n + z_n)).
\end{align*}

Since addition in $\F$ is associative, for each $j$:
\[
    (x_j + y_j) + z_j = x_j + (y_j + z_j).
\]

Therefore $(x + y) + z = x + (y + z)$. \hfill $\square$
\end{solution}

%------------------------------------------------------------------------------

\exercise
Show that $(ab)x = a(bx)$ for all $x \in \F^n$ and all $a, b \in \F$.

\begin{solution}
Let $x = (x_1, \ldots, x_n)$ where $x_j \in \F$.

Computing the left-hand side:
\[
    (ab)x = ((ab)x_1, \ldots, (ab)x_n).
\]

Computing the right-hand side:
\[
    a(bx) = a(bx_1, \ldots, bx_n) = (a(bx_1), \ldots, a(bx_n)).
\]

Since multiplication in $\F$ is associative, for each $j$:
\[
    (ab)x_j = a(bx_j).
\]

Therefore $(ab)x = a(bx)$. \hfill $\square$
\end{solution}

\newpage

%------------------------------------------------------------------------------

\exercise
Show that $1x = x$ for all $x \in \F^n$.

\begin{solution}
Let $x = (x_1, \ldots, x_n)$ where $x_j \in \F$.

By the definition of scalar multiplication:
\[
    1x = (1 \cdot x_1, \ldots, 1 \cdot x_n).
\]

Since $1$ is the multiplicative identity in $\F$, we have $1 \cdot x_j = x_j$ for each $j$.

Therefore:
\[
    1x = (x_1, \ldots, x_n) = x. \quad \square
\]
\end{solution}

%------------------------------------------------------------------------------

\exercise
Show that $\lambda(x + y) = \lambda x + \lambda y$ for all $\lambda \in \F$ and all $x, y \in \F^n$.

\begin{solution}
Let $x = (x_1, \ldots, x_n)$ and $y = (y_1, \ldots, y_n)$ where $x_j, y_j \in \F$.

Computing the left-hand side:
\begin{align*}
    \lambda(x + y) &= \lambda(x_1 + y_1, \ldots, x_n + y_n) \\
    &= (\lambda(x_1 + y_1), \ldots, \lambda(x_n + y_n)).
\end{align*}

Computing the right-hand side:
\begin{align*}
    \lambda x + \lambda y &= (\lambda x_1, \ldots, \lambda x_n) + (\lambda y_1, \ldots, \lambda y_n) \\
    &= (\lambda x_1 + \lambda y_1, \ldots, \lambda x_n + \lambda y_n).
\end{align*}

Since multiplication distributes over addition in $\F$, for each $j$:
\[
    \lambda(x_j + y_j) = \lambda x_j + \lambda y_j.
\]

Therefore $\lambda(x + y) = \lambda x + \lambda y$. \hfill $\square$
\end{solution}

%------------------------------------------------------------------------------

\exercise
Show that $(a + b)x = ax + bx$ for all $a, b \in \F$ and all $x \in \F^n$.

\begin{solution}
Let $x = (x_1, \ldots, x_n)$ where $x_j \in \F$.

Computing the left-hand side:
\[
    (a + b)x = ((a + b)x_1, \ldots, (a + b)x_n).
\]

Computing the right-hand side:
\begin{align*}
    ax + bx &= (ax_1, \ldots, ax_n) + (bx_1, \ldots, bx_n) \\
    &= (ax_1 + bx_1, \ldots, ax_n + bx_n).
\end{align*}

Since multiplication distributes over addition in $\F$, for each $j$:
\[
    (a + b)x_j = ax_j + bx_j.
\]

Therefore $(a + b)x = ax + bx$. \hfill $\square$
\end{solution}

\vfill

\end{document}
