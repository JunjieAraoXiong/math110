%==============================================================================
% EXERCISES 1A WORKSHEET
% Linear Algebra Done Right (4th ed.) - Sheldon Axler
%==============================================================================

\documentclass[11pt]{article}
\usepackage[margin=1in]{geometry}

% Required packages
\usepackage{amsmath, amssymb, amsthm}
\usepackage{enumitem}
\usepackage{fancyhdr}
\usepackage{tikz}
\usepackage{xcolor}

% Custom commands (from style guide)
\newcommand{\R}{\mathbb{R}}
\newcommand{\C}{\mathbb{C}}
\newcommand{\F}{\mathbb{F}}
\newcommand{\Z}{\mathbb{Z}}

% Header/footer
\pagestyle{fancy}
\fancyhf{}
\fancyhead[L]{\small MATH 110}
\fancyhead[R]{\small Exercises 1A}
\fancyfoot[C]{\small\thepage}
\renewcommand{\headrulewidth}{0.4pt}

% Exercise counter and command
\newcounter{exercise}
\newcommand{\exercise}{\refstepcounter{exercise}\noindent\textbf{Exercise \theexercise.} }

% Ruled solution space with gray separator
\newcommand{\ruledspace}[1]{%
  \par\vspace{0.3em}%
  \noindent\textcolor{gray!30}{\hrule}%
  \vspace{#1}%
}

% Answer line for computation problems
\newcommand{\answerline}{%
  \vspace{0.5em}%
  \noindent\textbf{Answer:} \hrulefill%
  \vspace{1em}%
}

\begin{document}

\noindent
\begin{minipage}{\linewidth}
    \centering
    \textbf{\Large Exercises 1A: $\R^n$ and $\C^n$} \\[0.5em]
    \textit{Linear Algebra Done Right, 4th ed.}\\[0.3em]
    \hrule
\end{minipage}
\vspace{1.5em}

%==============================================================================
% COMPLEX NUMBER EXERCISES (1-6)
%==============================================================================

\exercise
Suppose $a$ and $b$ are real numbers, not both $0$. Find real numbers $c$ and $d$ such that
\[
    \frac{1}{a + bi} = c + di.
\]
\ruledspace{4cm}
\answerline

%------------------------------------------------------------------------------

\exercise
Show that
\[
    \frac{-1 + \sqrt{3}\,i}{2}
\]
is a cube root of $1$ (meaning that its cube equals $1$).
\ruledspace{5cm}

%------------------------------------------------------------------------------

\exercise
Find two distinct square roots of $i$.
\ruledspace{4cm}
\answerline

\newpage

%------------------------------------------------------------------------------

\exercise
Show that $\alpha + \beta = \beta + \alpha$ for all $\alpha, \beta \in \C$.
\ruledspace{6cm}

%------------------------------------------------------------------------------

\exercise
Show that $(\alpha + \beta) + \lambda = \alpha + (\beta + \lambda)$ for all $\alpha, \beta, \lambda \in \C$.
\ruledspace{6cm}

%------------------------------------------------------------------------------

\exercise
Show that $(\alpha\beta)\lambda = \alpha(\beta\lambda)$ for all $\alpha, \beta, \lambda \in \C$.
\ruledspace{6cm}

\newpage

%==============================================================================
% MORE COMPLEX NUMBER EXERCISES (7-8)
%==============================================================================

\exercise
Show that for every $\alpha \in \C$, there exists a unique $\beta \in \C$ such that $\alpha + \beta = 0$.
\ruledspace{7cm}

%------------------------------------------------------------------------------

\exercise
Show that for every $\alpha \in \C$ with $\alpha \neq 0$, there exists a unique $\beta \in \C$ such that $\alpha\beta = 1$.
\ruledspace{7cm}

\newpage

%==============================================================================
% VECTOR EXERCISES (9-10)
%==============================================================================

\exercise
Find $x \in \R^4$ such that
\[
    (4, -3, 1, 7) + 2x = (5, 9, -6, 8).
\]
\ruledspace{4cm}
\answerline

%------------------------------------------------------------------------------

\exercise
Explain why there does not exist $\lambda \in \C$ such that
\[
    \lambda(2 - 3i, 5 + 4i, -6 + 7i) = (12 - 5i, 7 + 22i, -32 - 9i).
\]
\ruledspace{7cm}

\newpage

%==============================================================================
% F^n PROPERTIES (11-15)
%==============================================================================

\exercise
Show that $(x + y) + z = x + (y + z)$ for all $x, y, z \in \F^n$.
\ruledspace{6cm}

%------------------------------------------------------------------------------

\exercise
Show that $(ab)x = a(bx)$ for all $x \in \F^n$ and all $a, b \in \F$.
\ruledspace{6cm}

\newpage

%------------------------------------------------------------------------------

\exercise
Show that $1x = x$ for all $x \in \F^n$.
\ruledspace{5cm}

%------------------------------------------------------------------------------

\exercise
Show that $\lambda(x + y) = \lambda x + \lambda y$ for all $\lambda \in \F$ and all $x, y \in \F^n$.
\ruledspace{6cm}

%------------------------------------------------------------------------------

\exercise
Show that $(a + b)x = ax + bx$ for all $a, b \in \F$ and all $x \in \F^n$.
\ruledspace{6cm}

\vfill

\end{document}
